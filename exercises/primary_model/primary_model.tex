\documentclass[a4paper, notitlepage, 11pt]{article}

%----------------------------------------------
%-------------------Style----------------------
%----------------------------------------------

\usepackage{geometry}
\fontfamily{times}
\geometry{verbose,tmargin=30mm,bmargin=25mm,lmargin=25mm,rmargin=25mm}

%----------------------------------------------
%-----------------Packages---------------------
%----------------------------------------------

% Writing 
\usepackage[english]{babel}
\usepackage[utf8]{inputenc}
\usepackage[pdftex]{lscape}
\fontfamily{times}
\usepackage[affil-it]{authblk}
\usepackage{lettrine}

% Graphics
\usepackage{caption}
\usepackage{subcaption}
\usepackage{graphicx}
\graphicspath{{img/}}

% Math
\usepackage{amsthm, amssymb, amsmath, mathtools}

% Biblio
\usepackage[authoryear, round]{natbib}
\bibliographystyle{apalike}

%Others
\usepackage{url, hyperref}
\hypersetup{colorlinks=true,citecolor=blue}

%----------------------------------------------
%---------------Math definitions---------------
%----------------------------------------------

\newcommand{\R}{\mathbb{R}}

\newtheorem{theorem}{Theorem}[]
\newtheorem{proposition}{Proposition}[]

\theoremstyle{definition}
\newtheorem{definition}{Definition}[section]

\theoremstyle{remark}
\newtheorem*{remark}{Remark}
\newtheorem{assumption}{Assumption}

%----------------------------------------------
%-----------------Title Page-------------------
%----------------------------------------------

\title{Prevalence estimation}

\author{Lucas Machado Moschen}
\affil{School of Applied Mathematics, \\ Fundação Getulio Vargas}

\date{\today}

%----------------------------------------------
%------------------Document--------------------
%----------------------------------------------

\begin{document}
\maketitle

\section{Introduction}

A key question for epidemiologists and public health authorities is about the
proportion of individuals exposed to the disease at time $t$. This quantity
can be measured periodically, and the evolution shows how the transmission is
going on. For instance, if after a year the proportion grew 50\% it would be
worrisome. We call it prevalence. High prevalence of a disease within a
population might mean that there is a high incidence of it or prolonged
survival without cure. 

This report is the initial model for my bachelor dissertation entitled ``Bayesian analysis of respondent-driven surveys with
outcome uncertainty'', which proposes to study prevalence when the diagnostic
tests are imperfect and the population is hidden, that is, there is no
sampling frame for it \cite{heckathorn1997}. 

\section{Preliminary definitions}

Suppose we have a sample indexed by $i$. Let $X_i$ be the indicator
function of the $i^{th}$ individual exposed to the disease, and $T_i$ indicating whether the test in the $i^{th}$ individual is positive. Suppose that
$\{X_i\}$ and $\{T_i\}$ are two independent and identically distributed
random variables with $\Pr(X = 1) = \theta$ and $\Pr(T = 1) = p$. We say that
$\theta$ is the prevalence and $p$ is the apparent prevalence in the
population. 

If the test is perfect, $T_i = X_i$ for every $i$, and
$\theta = p$ (with probability one when they are random variables).
Unfortunately, this is not true in the real world. For that, the evaluation of
the diagnostic test must be regarded, and the following definitions are
important:

\begin{definition}[Specificity]
  Probability of a negative test correctly identified. In mathematical terms,
  conditioned on $X = 0$, the {\em specificity} $\gamma_e$ is the probability of $T = 0$: 
  \begin{equation}
    \gamma_e = \Pr(T = 0|X = 0). 
  \end{equation} 
\end{definition}

\begin{definition}[Sensitivity]
  Probability of a positive test correctly identified. In mathematical terms,
  conditioned on $X = 1$, the {\em sensitivity} $\gamma_s$ is the probability of $T = 1$: 
  \begin{equation}
    \gamma_s = \Pr(T = 1|X = 1). 
  \end{equation} 
\end{definition}

\begin{theorem}[Relation between prevalence and apparent prevalence] These quantities are related by the following equation:

  \begin{equation}
    p = \gamma_s\theta + (1-\gamma_e)(1-\theta).
  \end{equation}
  
\end{theorem}

\begin{proof}
  This is a direct application of the definition of conditional probability
  and the countable additivity axiom of Probability:
  \begin{equation*}
    \begin{split}
      p &= \Pr(T = 1) = \Pr(T = 1, R = 1) + \Pr(T = 1, R = 0) \\
      &= \Pr(T=1|R=1)\Pr(R=1) + \Pr(T=1|R=0)P(R=0) \\
      &= \Pr(T=1|R=1)\Pr(R=1) + (1 - \Pr(T=0|R=0))(1-P(R=1)) \\
      &= \gamma_s\theta + (1 - \gamma_e)(1-\theta)
    \end{split}
  \end{equation*} 
\end{proof}


\section{Prevalence model}



\begin{assumption}

\end{assumption}







\bibliography{biblio} 

\end{document}          
