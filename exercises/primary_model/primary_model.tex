\documentclass[a4paper, notitlepage, 11pt]{article}

%----------------------------------------------
%-------------------Style----------------------
%----------------------------------------------

\usepackage{geometry}
\fontfamily{times}
\geometry{verbose,tmargin=30mm,bmargin=25mm,lmargin=25mm,rmargin=25mm}

%----------------------------------------------
%-----------------Packages---------------------
%----------------------------------------------

% Writing 
\usepackage[english]{babel}
\usepackage[utf8]{inputenc}
\usepackage[pdftex]{lscape}
\fontfamily{times}
\usepackage[affil-it]{authblk}
\usepackage{lettrine}

% Graphics
\usepackage{caption}
\usepackage{subcaption}
\usepackage{graphicx}
\graphicspath{{img/}}

% Math
\usepackage{amsthm, amssymb, amsmath, mathtools}

% Biblio
\usepackage[authoryear, round]{natbib}
\bibliographystyle{apalike}

%Others
\usepackage{url, hyperref}
\hypersetup{colorlinks=true,citecolor=blue}

%----------------------------------------------
%---------------Math definitions---------------
%----------------------------------------------

\newcommand{\R}{\mathbb{R}}
\newcommand{\x}{\boldsymbol{x}}

\newtheorem{theorem}{Theorem}[]
\newtheorem{proposition}{Proposition}[]

\theoremstyle{definition}
\newtheorem{definition}{Definition}[section]

\theoremstyle{remark}
\newtheorem*{remark}{Remark}
\newtheorem{assumption}{Assumption}

%----------------------------------------------
%-----------------Title Page-------------------
%----------------------------------------------

\title{Prevalence estimation}

\author{Lucas Machado Moschen}
\affil{School of Applied Mathematics, \\ Fundação Getulio Vargas}

\date{\today}

%----------------------------------------------
%------------------Document--------------------
%----------------------------------------------

\begin{document}
\maketitle

\section{Introduction}

A key question for epidemiologists and public health authorities is about the
proportion of individuals exposed to the disease at time $t$. This quantity
can be measured periodically, and the evolution shows how the transmission is
going on. For instance, if after a year the proportion grew 50\% it would be
worrisome. We call it prevalence. High prevalence of a disease within a
population might mean that there is a high incidence of it or prolonged
survival without cure. 

This report is the initial model for my bachelor dissertation entitled ``Bayesian analysis of respondent-driven surveys with
outcome uncertainty'', which proposes to study prevalence when the diagnostic
tests are imperfect and the population is hidden, that is, there is no
sampling frame for it \cite{heckathorn1997}. 

\section{Preliminary definitions}

Suppose we have a sample indexed by $i$. Let $Y_i$ be the indicator
function of the $i^{th}$ individual exposed to the disease, and $T_i$ indicating whether the test in the $i^{th}$ individual is positive. Suppose that
$\{Y_i\}$ and $\{T_i\}$ are two independent and identically distributed
random variables with $\Pr(X = 1) = \theta$ and $\Pr(T = 1) = p$. We say that
$\theta$ is the prevalence and $p$ is the apparent prevalence in the
population. 

If the test is perfect, $T_i = Y_i$ for every $i$, and
$\theta = p$ (with probability one when they are random variables).
Unfortunately, this is not true in the real world. For that, the evaluation of
the diagnostic test must be regarded, and the following definitions are
important:

\begin{definition}[Specificity]
  Probability of a negative test correctly identified. In mathematical terms,
  conditioned on $Y = 0$, the {\em specificity} $\gamma_e$ is the probability of $T = 0$: 
  \begin{equation}
    \gamma_e = \Pr(T = 0|Y = 0). 
  \end{equation} 
\end{definition}

\begin{definition}[Sensitivity]
  Probability of a positive test correctly identified. In mathematical terms,
  conditioned on $Y = 1$, the {\em sensitivity} $\gamma_s$ is the probability of $T = 1$: 
  \begin{equation}
    \gamma_s = \Pr(T = 1|Y = 1). 
  \end{equation} 
\end{definition}

\begin{theorem}[Relation between prevalence and apparent prevalence] These quantities are related by the following equation:

  \begin{equation}
    p = \gamma_s\theta + (1-\gamma_e)(1-\theta).
  \end{equation}
  
\end{theorem}

\begin{proof}
  This is a direct application of the definition of conditional probability
  and the countable additivity axiom of Probability:
  \begin{equation*}
    \begin{split}
      p &= \Pr(T = 1) = \Pr(T = 1, Y = 1) + \Pr(T = 1, Y = 0) \\
      &= \Pr(T=1|Y=1)\Pr(Y=1) + \Pr(T=1|Y=0)\Pr(Y=0) \\
      &= \Pr(T=1|Y=1)\Pr(Y=1) + (1 - \Pr(T=0|Y=0))(1-\Pr(Y=1)) \\
      &= \gamma_s\theta + (1 - \gamma_e)(1-\theta)
    \end{split}
  \end{equation*} 
\end{proof}

The intuition behind this equation is pretty simple: the proportion
of positive test counts the correct identified exposed individuals and the
incorrect identified not exposed. Observe that if $\gamma_s = \gamma_e = 1$, we have the trivial case $p =
\theta$. Moreover, if $\gamma_s = \gamma_e = 0.5$, we have that
$p = 0.5$ and it is impossible to have information about $\theta$. 

\begin{remark}
  Actually, we are interested in the pontual prevalence at time $t$. Being
  impossible to test every individual at the same time, we assume that all
  individuals remain exposed to the disease at time of the last tested. 
\end{remark}

\section{Prevalence model}

Firstly, we make some assumptions to simplify the modeling:

\begin{assumption}
  For a Bayesian modeling, we assume each parameter of interest of the model
  has a probability distribution to incorporate the uncertainty about it. 
\end{assumption}

\begin{assumption}
  Suppose for each individual we observe $k$ features that can be possible
  risk factors, and $\x_i$ is the vector for the $i^{th}$ individual.
\end{assumption}

\begin{assumption}
  Suppose each individual has a probability $\theta_i$ of having been exposed
  to the disease and $\theta_i$ dependes on $\x_i$, not necessarily linearly.
  We therefore will have $p_i$ the probability of positive test in the
  $i^{th}$ individual.
\end{assumption} 

\begin{assumption}
  We assume sensitivity and specificity have the same distribution for all
  individuals and it only depends on the test used to diagnose. 
\end{assumption}

From above, we develop three different models with different stages: 

\subsection{Perfect tests}

The first model only consider the risk factors $\x_i$. 

\begin{equation}
  \begin{aligned}
    Y_i &\sim \operatorname{Bernoulli}(\theta_i) \\
    g(\theta_i) &= \x_i\beta, 
  \end{aligned}  
\end{equation}
where $g(\cdot)$ is a link function. These class of functions maps a
non-linear relationship to a linear one. Examples include the logit and probit
functions. For a Bayesian inference, priors on $\beta$ must be included. 



\bibliography{biblio} 

\end{document}          
