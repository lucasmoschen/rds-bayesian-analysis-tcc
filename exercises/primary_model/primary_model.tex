\documentclass[a4paper, notitlepage, 11pt]{article}

%----------------------------------------------
%-------------------Style----------------------
%----------------------------------------------

\usepackage{geometry}
\fontfamily{times}
\geometry{verbose,tmargin=20mm,bmargin=25mm,lmargin=25mm,rmargin=25mm}

%----------------------------------------------
%-----------------Packages---------------------
%----------------------------------------------

% Writing 
\usepackage[english]{babel}
\usepackage[utf8]{inputenc}
\usepackage[pdftex]{lscape}
\fontfamily{times}
\usepackage[affil-it]{authblk}
\usepackage{lettrine}

% Graphics
\usepackage{caption}
\usepackage{subcaption}
\usepackage{graphicx}
\graphicspath{{img/}}

% Math
\usepackage{amsthm, amssymb, amsmath, mathtools}

% Biblio
\usepackage[authoryear, round]{natbib}
\bibliographystyle{apalike}

%Others
\usepackage{url, hyperref}
\hypersetup{colorlinks=true,citecolor=blue}

%----------------------------------------------
%---------------Math definitions---------------
%----------------------------------------------

\newcommand{\R}{\mathbb{R}}
\newcommand{\x}{\boldsymbol{x}}
\newcommand{\N}{\operatorname{Normal}}
\newcommand{\betadist}{\operatorname{Beta}}
\newcommand{\tril}{\operatorname{tril}}

\newtheorem{theorem}{Theorem}[]
\newtheorem{proposition}{Proposition}[]

\theoremstyle{definition}
\newtheorem{definition}{Definition}[section]

\theoremstyle{remark}
\newtheorem*{remark}{Remark}
\newtheorem{assumption}{Assumption}

\newcommand{\improve}[1]{\textcolor{red}{#1}}

%----------------------------------------------
%-----------------Title Page-------------------
%----------------------------------------------

\title{Prevalence estimation}

\author{Lucas Machado Moschen}
\affil{School of Applied Mathematics, \\ Fundação Getulio Vargas}

\date{\today}

%----------------------------------------------
%------------------Document--------------------
%----------------------------------------------

\begin{document}
\maketitle

\section{Introduction}

A key quantity for epidemiologists and public health researchers is the
proportion of individuals exposed to a disease at time $t$, which is called
{\em prevalence}. When measured
periodically, its evolution can identify potential causes of the infection
and prevention and care methods \cite[]{noordzij2010measures}. The prevalence
differs from {\em incidence } that measures the proportion of people who
develop new disease during a specified period of time
\cite[]{rothman2008modern}. Therefore, prevalence reflects both incidence and the
duration of disease. 

This report presents the initial models for my bachelor dissertation entitled ``Bayesian analysis of respondent-driven surveys with
outcome uncertainty'', which proposes to study prevalence when the diagnostic
tests are imperfect and the population is hidden, that is, there is no
sampling frame for it \cite[]{heckathorn1997}. 

\subsection{Respondent-driven sampling}

Respondent-driven sampling (RDS) is commonly used to survey hidden or hard-to-reach populations when
no sampling frame exists \cite[]{heckathorn1997}, which means there is no
enumeration of the population, since size and boundaries are unknown. In this approach, the
researchers select some individuals, called {\em seeds} from the target
population, and give them a fixed amount of {\em recruitment coupons} to
recruit their peers. Each recipient of the coupons reclaims it in the study
site, is interviewed, and receives more coupons to continue the recruitment.
This process occurs until some criteria is reached. The sampling is without
replacement, so the participants cannot be recruited more than once. Moreover,
the respondents inform how many subjects from the population they know.

The subjects receive a reward for being interviewed and for each recruitment
of their peers which establishes a dual incentive system. The {\em primary incentive} is the
{\em individual-sanction-based control}, so there is a reward for
participating. The second one is the {\em group-mediated social control} that
influences the participants to induce others to comply to get the reward for the recruitment. When social approval is important, recruitment can be even
more efficient and cheaper, since material incentive can be converted into
symbolic by the individuals. In summary, accepting to be recruited will have a
material incentive for both and a symbolic incentive for the recruited, since
theirs peers also participated.

Let $G = (V,E)$ be an undirected graph representing the hidden population. The {\em recruitment graph} $G_R =
(V_R, E_R)$ represents the recruited individuals and the recruitment edges,
that is, $(i,j) \in E_R$ if, and only if, $i$ recruited $j$.
Given that each individual can be sampled only once, it is not possible to
observe the {\em recruitment-induced subgraph}, that is the induced subgraph
generated by $V_R$. Moreover, the {\em coupon matrix} $C$ defined by $C_{ij} =
1$ if the i$^{th}$ subject has at least one coupon before the j$^{th}$
recruitment event, is also observed with the recruitment times. Assuming an
exponential and independent distribution of the times, the likelihood can be
written explicitly, and the distribution interpreted as an exponential random graph
model \cite[]{crawford2016}.  

These models allowed several applications in social sciences, epidemiology,
and statistics, including hidden populations size estimation
\cite[]{crawford2018hidden}, regression \cite[]{bastos2012binary}, communicable
disease prevalence estimation \cite[]{albuquerque2009avaliaccao}, among others.

\section{Preliminary definitions}

Let $I$ be a index set and $Y_i$ be the indicator function of the $i^{th}$ individual's exposure to the disease, and $T_i$
indicating whether the test of the $i^{th}$ individual is positive at time
$t$. Suppose that $\{Y_i\}_{i \in I}$ and $\{T_i\}_{i \in I}$ are two independent and identically distributed
random variables with $\Pr(X = 1) = \theta$ and $\Pr(T = 1) = p$. We say that
$\theta$ is the prevalence and $p$ is the apparent prevalence in the
population. 

If the test is perfect, then for every $i$, $T_i = Y_i$, and
$\theta = p$ (with probability one when they are random variables).
Unfortunately, this is not true in the real world, what makes important to
regard the evaluation of the diagnostic, and the following definitions are used:

\begin{definition}[Specificity]
  Probability of a negative test correctly identified. In mathematical terms,
  conditioned on $Y = 0$, the {\em specificity} $\gamma_e$ is the probability of $T = 0$: 
  \begin{equation}
    \gamma_e = \Pr(T = 0|Y = 0). 
  \end{equation} 
\end{definition}

\begin{definition}[Sensitivity]
  Probability of a positive test correctly identified. In mathematical terms,
  conditioned on $Y = 1$, the {\em sensitivity} $\gamma_s$ is the probability of $T = 1$: 
  \begin{equation}
    \gamma_s = \Pr(T = 1|Y = 1). 
  \end{equation} 
\end{definition}

\begin{theorem}[Relation between prevalence and apparent prevalence] These quantities are related by the following equation:
  \begin{equation}
    p = \gamma_s\theta + (1-\gamma_e)(1-\theta).
  \end{equation}
  
\end{theorem}

\begin{proof}
  This is a direct application of the definition of conditional probability
  and the countable additivity axiom of Probability:
  \begin{equation*}
    \begin{split}
      p &= \Pr(T = 1) = \Pr(T = 1, Y = 1) + \Pr(T = 1, Y = 0) \\
      &= \Pr(T=1|Y=1)\Pr(Y=1) + \Pr(T=1|Y=0)\Pr(Y=0) \\
      &= \Pr(T=1|Y=1)\Pr(Y=1) + (1 - \Pr(T=0|Y=0))(1-\Pr(Y=1)) \\
      &= \gamma_s\theta + (1 - \gamma_e)(1-\theta).
    \end{split}
  \end{equation*} 
\end{proof}

The intuition behind this equation is pretty simple: the proportion
of positive test counts the correct identified exposed individuals and the
incorrect identified not exposed. Observe that if $\gamma_s = \gamma_e = 1$, we have the trivial case $p =
\theta$. Moreover, if $\gamma_s = \gamma_e = 0.5$, we have that
$p = 0.5$ and there is no information about $\theta$. 

\begin{remark}
  Actually, we are interested in the prevalence at time $t$. When it is 
  impossible to test every individual at the same time, we assume that all
  individuals remain exposed to the disease at time of the last tested individual. 
\end{remark}

\improve{\begin{definition}[Link function]
  A class of functions which maps a non-linear relationship to a linear one.
  Here we consider functions with domain $[0,1]$. Examples include the logit
  and probit functions.
\end{definition}}

\section{Prevalence model}

Firstly, we make some assumptions to simplify the modeling:

\begin{assumption}  
  For a Bayesian modeling, we assume each model's parameter has a probability distribution that incorporates the researcher's uncertainty about it. 
\end{assumption}

\begin{assumption}
  For each individual, we observe $k$ \improve{regressors} that are possible
  risk factors represented by the vector $\x_i \in \R^{k}$ of the $i^{th}$
  individual. We assume that the probability $\theta_i$ of the $i^{th}$ individual having been exposed
  to the disease dependes on the prevalence $\theta$ and $\x_i$. The
  probability of positive test in the $i^{th}$ individual is denoted by $p_i$. Therefore, the sequences $\{Y_i\}_{i \in I}$ and $\{T_i\}_{i \in I}$ are not
  identically distributed anymore.
\end{assumption}

\begin{assumption}
  Sensitivity and specificity have the same distribution for all
  individuals and it only depends on the test used to diagnose. 
\end{assumption}

From above, we develop three different models.

\subsection{Perfect tests}

The first model supposes the samples are independent and the test is perfect,
which means that $\theta_i = p_i$ for all $i$. Therefore it only considers the risk factors $\x_i$. 

\begin{equation}
  \begin{aligned}
    T_i &\sim \operatorname{Bernoulli}(\theta_i), \\
    g(\theta_i) &= g(\theta) + \x_i^T\beta, 
  \end{aligned}  
\end{equation}
where $v^T$ denotes the transpose of $v$, and $g(\cdot)$ is a link function.
The parameter $\beta \in \R^{k}$ is the risk effects. For Bayesian inference, priors on
$\beta$ and $\theta$ must be included. We use $\beta ~ \sim \N(\mu, \Sigma)$
and $\theta \sim \betadist(a^{p}, b^p)$, where $\mu
\in \R^{k}$, $\Sigma \in \R^{k\times k}$ symmetric positive-definite matrix,
$a^p \in \R_{++}$, and $b^p \in \R_{++}$
are fixed hyperparameters. 

\begin{remark}
  If the risk factors are zero, i.e $\x_i = 0$, the probability of the
  $i^{th}$ having been exposed is the prevalence $\theta$, which means that in
  a population with no risk effects, the probability of a person has the
  disease is exactly the proportion in this population. 
\end{remark}

\subsubsection{Identifiability}

\subsubsection{Experiments}

\url{https://github.com/lucasmoschen/rds-bayesian-analysis/blob/main/exercises/primary_model/model_experiments.ipynb}

\subsection{Imperfect tests}

This model includes the sensitivity and specificity of the diagnostic test. 

\begin{equation}
  \begin{aligned}
    T_i &\sim \operatorname{Bernoulli}(p_i) \\
    p_i &= \gamma_s\theta_i + (1-\gamma_e)(1 - \theta_i),  \\
    g(\theta_i) &= g(\theta) + \x_i^T\beta,  \\
    \beta &\sim \N(\mu, \Sigma), \\ 
    \theta &\sim \betadist(a^p, b^p) \\
    \gamma_s &\sim \betadist(a^s, b^s), \\
    \gamma_e &\sim \betadist(a^e, b^e), \\    
  \end{aligned}  
\end{equation}
where $a^p, a^s, a^e, b^p, b^s, b^e \in \R_{++}$ are fixed hyperparameters.
This model does not include prior knowledge about the correlation between
specificity and sensitivity. 

\subsection{Imperfect test and respondent-driven sampling}

For now, we consider the network dependence induced by the RDS. Ideally, we 
should build a model for it and include the associated uncertainty.
Since it is not possible (for a while), we treat it as a random effect for
each individual. Conditionally autoregressive (CAR) models in the
Gaussian case are used. Therefore: 
\begin{equation}
  \begin{aligned}
    T_i &\sim \operatorname{Bernoulli}(p_i) \\
    p_i &= \gamma_s\theta_i + (1-\gamma_e)(1 - \theta_i),  \\
    g(\theta_i) &= g(\theta) + \x_i^T\beta + \omega_i,  \\
    \omega_i|\{\omega_j\}_{j\neq i}, \tau &\sim \N\left(\rho\sum_j q_{ij}\omega_j/q_{i+}, \tau^2/q_{i+}\right) \\
    \beta &\sim \N(\mu, \Sigma), \\ 
    \theta &\sim \betadist(a^p, b^p) \\
    \gamma_s &\sim \betadist(a^s, b^s), \\
    \gamma_e &\sim \betadist(a^e, b^e), \\  
    \tau &\sim \N^+(0,\sigma^2_{\tau}),  
  \end{aligned}  
\end{equation}
such that $\rho$ is a fixed parameter in order to the distribution be proper, $[Q]_{ij} = q_{ij}$ is a fixed matrix to measure the distance between $i$
and $j$, and $q_{i+} = \sum_{j} q_{ij}$. For instance,
$$
q_{ij} = \begin{cases}
  1, &\text{if } i \text{ recruited } j \text{ or the contrary} \\
  0, &\text{otherwise.} 
\end{cases}
$$

\subsubsection{Exponential Random Graph Model (ERGM)}

RDS has the constraint of being without replacement. For that reason, we do
not observe all links among the samples \cite[]{crawford2016}. Considering the
model developed by Crawford, we can model the
matrix $Q$ as {\em Exponential Random Graph Model} (ERGM). Define the
following 

\begin{enumerate}
  \item $\boldsymbol{s} = \tril(QC)^T \boldsymbol{1} + C^Tu$, such that $Q$ is the
  adjacency matrix of the recruited subjects, $C$ is the {\em Coupon Matrix},
  $u$ the vector of the number of edge ends belonging to each vertex
  (in the order of recruitment) that are not connected to any other sampled
  vertex, and $\tril(M)$ the lower triangle of $M$. 

  \item $T(Q) = -\lambda \boldsymbol{s}$, such that $\lambda$ is the rate of
  the recruitment time. 

  \item $V(Q) = \sum_{k \text{ is not seed}} \log(\lambda \boldsymbol{s}_k)$
  
  \item $w = (0, t_2 - t_1, ..., t_n - t_{n-1})$ is the vector of the waiting times between
  recruitments.  
\end{enumerate}

Therefore $\Pr(Q|w) \propto \exp[T(Q)^Tw + V(Q)]$. With that, the model
becomes 

\begin{equation}
  \begin{aligned}
    T_i &\sim \operatorname{Bernoulli}(p_i) \\
    p_i &= \gamma_s\theta_i + (1-\gamma_e)(1 - \theta_i),  \\
    g(\theta_i) &= g(\theta) + \x_i^T\beta + \omega_i,  \\
    \omega_i|\{\omega_j\}_{j\neq i}, \tau &\sim \N\left(\rho\sum_j q_{ij}\omega_j/q_{i+}, \tau^2/q_{i+}\right) \\
    Q|w &\propto \exp[T(Q)^Tw + V(Q)] \\
    \lambda &\sim \Gamma(a^{\lambda}, b^{\lambda}), \\ 
    \beta &\sim \N(\mu, \Sigma), \\ 
    \theta &\sim \betadist(a^p, b^p) \\
    \gamma_s &\sim \betadist(a^s, b^s), \\
    \gamma_e &\sim \betadist(a^e, b^e), \\  
    \tau &\sim \N^+(0,\sigma^2_{\tau}).
  \end{aligned}  
\end{equation}
The problem with this model is that we are assigning a posterior distribution
for $Q$.

\section{Correlation analysis between specificity and sensitivity}

In this section, we shall describe how to use the Bivariate Beta
\cite[]{olkin2015constructions} to model the correlation between specificity
and sensitivity.

\subsection{Technical details}

Let $U = (U_1, U_2, U_3, U_4) \sim
\operatorname{Dirichlet}(\boldsymbol{\alpha})$, where $\boldsymbol{\alpha} =
(\alpha_1, \alpha_2, \alpha_3, \alpha_4)$ and $U_4 = 1 - U_1 + U_2 + U_3$.
Therefore the joint density of $U$ is 
\begin{equation}
  f_U(u_1, u_2, u_3) = \frac{1}{B(\alpha_1,\alpha_2,\alpha_3,\alpha_4)}u_1^{\alpha_1-1}u_2^{\alpha_2-1}u_3^{\alpha_3-1}(1-u_1-u_2-u_3)^{\alpha_4-1}, 
\end{equation}
where $u_i \in [0,1], i = 1,2,3$, $u_1 + u_2 + u_3 \le 1$, $\alpha_i > 0, i =
1,...,4$ and for $v \in \R^n$,
$$B(v) = \frac{\prod_{i=1}^n \Gamma(v_i)}{\Gamma\left(\sum_{i=1}^n v_i\right)}.$$ 
Define
\begin{equation}
  X = U_1 + U_2 \text{ and } Y = U_1 + U_3.
\end{equation} 
Let's check the distribution of $X$ first. For that, we need the distribution
of $(U_1, U_2)$. 
\begin{equation}
  \label{eq:dist-u1-u2}
  \begin{split}
    f_{U_1, U_2}(u_1, u_2) &= \int_{-\infty}^{\infty} f_{U}(u_1,u_2,u_3) \, du_3 \\ 
    &= \frac{1}{B(\boldsymbol{\alpha})}\int_0^1 u_1^{\alpha_1-1}u_2^{\alpha_2-1}u_3^{\alpha_3-1}(1-u_1-u_2-u_3)^{\alpha_4-1} \, du_3 \\
    &= \frac{1}{B(\boldsymbol{\alpha})}u_1^{\alpha_1-1}u_2^{\alpha_2-1}\int_0^1 u_3^{\alpha_3-1}(1-u_1-u_2-u_3)^{\alpha_4-1} \, du_3.
  \end{split}
\end{equation}
Let $u_3 = (1 - u_1 - u_2)z$. Then,
\begin{equation}
  \begin{split}
    f_{U_1, U_2}(u_1, u_2) &= \frac{1}{B(\boldsymbol{\alpha})}u_1^{\alpha_1-1}u_2^{\alpha_2-1}\int_0^1 (1-u_1-u_2)^{\alpha_3-1}z^{\alpha_3-1}(1-u_1-u_2)^{\alpha_4}(1-z)^{\alpha_4-1} \, dz. \\
    &= \frac{1}{B(\boldsymbol{\alpha})}u_1^{\alpha_1-1}u_2^{\alpha_2-1}(1-u_1-u_2)^{\alpha_3+\alpha_4-1}\int_0^1 z^{\alpha_3-1}(1-z)^{\alpha_4-1} \, dz. \\
    &= \frac{1}{B(\boldsymbol{\alpha})}u_1^{\alpha_1-1}u_2^{\alpha_2-1}(1-u_1-u_2)^{\alpha_3+\alpha_4-1}\frac{\Gamma(\alpha_3)\Gamma(\alpha_4)}{\Gamma(\alpha_3 + \alpha_4)} \\
    &= \frac{1}{B(\alpha_1, \alpha_2, \alpha_3+\alpha_4)}u_1^{\alpha_1-1}u_2^{\alpha_2-1}(1-u_1-u_2)^{\alpha_3+\alpha_4-1}.
  \end{split}
\end{equation}
As expected, $(U_1, U_2, 1-U_1-U_2) \sim
\operatorname{Dirichlet}(\alpha_1,\alpha_2,\alpha_3+\alpha_4)$. Define 
$$
H(v) = \begin{bmatrix}
  1 & 0 \\ 1 & 1
\end{bmatrix}v, \text{ for } v \in \R^2.
$$
Then $(U_1, X) = H(U_1, U_2)$. By the Change of variables theorem, 
\begin{equation}
  \begin{split}
    f_{U_1, X}(u_1, x) &= f({H^{-1}(u_1,x)})\bigg|\det\left[\frac{dH^{-1}(v)}{dv}\bigg|_{v=(u_1,x)}\right]\bigg| \\ 
    &= f(u_1, x - u_1) = \frac{1}{B(\alpha_1, \alpha_2, \alpha_3+\alpha_4)}u_1^{\alpha_1-1}(x-u_1)^{\alpha_2-1}(1-x)^{\alpha_3+\alpha_4-1}, 
  \end{split}
\end{equation}
where $(u_1, x)$ belongs to the triangle defined by the points (0,0),
(0,1), and (1,1). The distribution of $X$ for $x \in [0,1]$ is
\begin{equation}
  \begin{split}
    f_X(x) &= \frac{1}{B(\alpha_1, \alpha_2, \alpha_3+\alpha_4)}\int_{0}^{x} u_1^{\alpha_1-1}(x-u_1)^{\alpha_2-1}(1-x)^{\alpha_3+\alpha_4-1} \, du_1 \\
    &= \frac{1}{B(\alpha_1, \alpha_2, \alpha_3+\alpha_4)}(1-x)^{\alpha_3+\alpha_4-1} \int_{0}^{x} u_1^{\alpha_1-1}(x-u_1)^{\alpha_2-1} \, du_1. \\
    &= \frac{1}{B(\alpha_1, \alpha_2, \alpha_3+\alpha_4)}(1-x)^{\alpha_3+\alpha_4-1} \int_{0}^{x} x^{\alpha_1-1} \left(\frac{u_1}{x}\right)^{\alpha_1-1}x^{\alpha_2 - 1}\left(1-\frac{u_1}{x}\right)^{\alpha_2-1} \, du_1. \\
  \end{split}
\end{equation}
Setting $u = u_1/x$ (if $x = 0, f_X(x) = 0$, then suppose $x > 0$), we have, 
\begin{equation}
  \begin{split}
    f_X(x) &= \frac{1}{B(\alpha_1, \alpha_2, \alpha_3+\alpha_4)}(1-x)^{\alpha_3+\alpha_4-1} x^{\alpha_1+\alpha_2-1} \int_{0}^{1} u^{\alpha_1-1}(1-u)^{\alpha_2-1} \, du. \\
    &= \frac{1}{B(\alpha_1, \alpha_2, \alpha_3+\alpha_4)}(1-x)^{\alpha_3+\alpha_4-1} x^{\alpha_1+\alpha_2-1} B(\alpha_1, \alpha_2)\\
    &= \frac{1}{B(\alpha_1 + \alpha_2, \alpha_3+\alpha_4)}(1-x)^{\alpha_3+\alpha_4-1} x^{\alpha_1+\alpha_2-1}\\
  \end{split}
\end{equation}
Therefore $X \sim \betadist(\alpha_1+\alpha_2, \alpha_3+\alpha_4)$.
Equivalently, $Y \sim \betadist(\alpha_1+\alpha_3, \alpha_2 + \alpha_4)$.
Moreover the joint distribution of $X$ and $Y$ is 
\begin{equation}
  f_{X,Y}(x,y) = \frac{1}{B(\boldsymbol{\alpha})}\int_{\Omega} u_1^{\alpha_1 - 1}(x - u_1)^{\alpha_2 -1}(y-u_1)^{\alpha_3-1}(1-x-y+u_1)^{\alpha_4-1} \, du_1,
\end{equation}
where 
$$
\Omega = (\max(0, x+y-1), \min(x,y)).
$$

Integrating $u_2$ out of equation \eqref{eq:dist-u1-u2}, we obtain that $U_1
\sim \betadist(\alpha_1, \alpha_2+\alpha_3+\alpha_4)$. Moreover,
$$U_i \sim \betadist(\alpha_i, \sum_{j\neq i} \alpha_i), i = 1,...,4.$$ 

Let $\tilde{a} = \sum_i \alpha_i$. The covariance of $U_i$ and $U_j$ is \cite[]{lin2016dirichlet} 
\begin{equation}
  \operatorname{Cov}(U_i, U_j) = - \frac{\alpha_i\alpha_j}{\tilde{\alpha}^2(\tilde{\alpha}+1)}, i,j = 1,...,4, i \neq j
\end{equation} 
and the variance of $U_i$ is 
\begin{equation}
  \operatorname{Var}(U_i) = \frac{\alpha_i(\tilde{\alpha}-\alpha_i)}{\tilde{\alpha}^2(\tilde{\alpha}+1)}
\end{equation}

Hence, 
\begin{align*}
    \operatorname{E}(X) &= \operatorname{E}(U_1 + U_2) = \frac{\alpha_1+\alpha_2}{\alpha_1+\alpha_2+\alpha_3+\alpha_4} \\
    \operatorname{E}(Y) &= \operatorname{E}(U_1 + U_3) = \frac{\alpha_1+\alpha_3}{\alpha_1+\alpha_2+\alpha_3+\alpha_4} \\
    \operatorname{Var}(X) &= \operatorname{Cov}(U_1+U_2, U_1+U_2) = \frac{1}{\tilde{\alpha}^2(\tilde{\alpha}+1)}(\alpha_1+\alpha_2)(\alpha_3 + \alpha_4) \\
    \operatorname{Var}(Y) &= \operatorname{Cov}(U_1+U_3, U_1+U_3) = \frac{1}{\tilde{\alpha}^2(\tilde{\alpha}+1)}(\alpha_1+\alpha_3)(\alpha_2 + \alpha_4)  \\ 
    \operatorname{Cov}(X,Y) &= \operatorname{Cov}(U_1+U_2, U_1+U_3) = \frac{1}{\tilde{\alpha}^2(\tilde{\alpha}+1)}(\alpha_1\alpha_4 - \alpha_2\alpha_3) \\
    \operatorname{Cor}(X,Y) &= \frac{\operatorname{Cov}(X,Y)}{\sqrt{\operatorname{Var}(X)\operatorname{Var}(Y)}} = \frac{\alpha_1\alpha_4 - \alpha_2\alpha_3}{\sqrt{(\alpha_1+\alpha_2)(\alpha_3+\alpha_4)(\alpha_1+\alpha_3)(\alpha_2+\alpha_4)}}
\end{align*}

The original paper has a mistake in page 6\footnote{\url{https://www.wolframalpha.com/input/?i=simplify+\%28a\%28a\%2B1\%29+\%2B+a*b+\%2B+a*c+\%2B+b*c\%29\%2F\%28\%28a\%2Bb\%2Bc\%2Bd\%29*\%28a\%2Bb\%2Bc\%2Bd\%2B1\%29\%29+-+\%28a+\%2B+b\%29*\%28a\%2Bc\%29\%2F\%28a\%2Bb\%2Bc\%2Bd\%29\%5E2+}}.


\bibliography{biblio} 

\end{document}          
