\documentclass{beamer}

%------------------------------------
%------------Libraries---------------
%------------------------------------

\usepackage[utf8]{inputenc}
\usepackage{xpatch}
\usepackage{xcolor}
\usepackage{url, hyperref}

\usepackage{amsmath, amsthm, amssymb, amsfonts} 

%------------------------------------
%----------Configurations------------
%------------------------------------

\usetheme{Madrid}
\usecolortheme{default}
\useinnertheme{circles}

\definecolor{FirstColor}{rgb}{0.0157,0.2392,0.4902}
\definecolor{SecondColor}{rgb}{0.0157, 0.549, 0.8}

\setbeamertemplate{itemize items}[triangle]

\setbeamercolor*{palette primary}{bg=FirstColor, fg=white}
\setbeamercolor*{palette secondary}{bg=SecondColor, fg=white}
\setbeamercolor*{palette tertiary}{bg=white, fg=FirstColor}
\setbeamercolor*{palette quaternary}{bg=FirstColor,fg=white}
\setbeamercolor{structure}{fg=FirstColor}
\setbeamercolor{section in toc}{fg=FirstColor}

\hypersetup{colorlinks=true,citecolor=blue, urlcolor = cien, linkcolor=blue}

%---------------------------------------------------
%------------------Itemize--------------------------
%---------------------------------------------------

\makeatletter
\newcommand{\my@beamer@setsep}{%
\ifnum\@itemdepth=1\relax
     \setlength\itemsep{\my@beamer@itemsepi}% separation for first level
   \else
     \ifnum\@itemdepth=2\relax
       \setlength\itemsep{\my@beamer@itemsepii}% separation for second level
     \else
       \ifnum\@itemdepth=3\relax
         \setlength\itemsep{\my@beamer@itemsepiii}% separation for third level
   \fi\fi\fi}
\newlength{\my@beamer@itemsepi}\setlength{\my@beamer@itemsepi}{3ex}
\newlength{\my@beamer@itemsepii}\setlength{\my@beamer@itemsepii}{1.5ex}
\newlength{\my@beamer@itemsepiii}\setlength{\my@beamer@itemsepiii}{1.5ex}
\newcommand\setlistsep[3]{%
    \setlength{\my@beamer@itemsepi}{#1}%
    \setlength{\my@beamer@itemsepii}{#2}%
    \setlength{\my@beamer@itemsepiii}{#3}%
}
\xpatchcmd{\itemize}
  {\def\makelabel}
  {\my@beamer@setsep\def\makelabel}
 {}
 {}

\xpatchcmd{\beamer@enum@}
  {\def\makelabel}
  {\my@beamer@setsep\def\makelabel}
 {}
 {}
\makeatother

%---------------------------------------------------
%-----------------Definitions-----------------------
%---------------------------------------------------

\newcommand{\Space}{\vspace{3ex}}

%---------------------------------------------------
%----------------Front page-------------------------
%---------------------------------------------------

\title[Respondent driven-sampling]
{Respondent driven-sampling}
\subtitle{Procedure to sample from hidden or hard-to-reach populations}
\author[Lucas Moschen]
{Lucas Moschen}
\institute[EMAp/FGV]
{
  School of Applied Mathematics\\
  Fundação Getulio Vargas
}
\date[\today]
{\today}

\titlegraphic{
    \vspace*{1.2cm}
    \hspace*{9.5cm}
    \includegraphics[width=.2\textwidth]{images/logo-emap.png}
}
%--------------------------------------------------

\AtBeginSection[]
{
  \begin{frame}
    \frametitle{Table of Contents}
    \tableofcontents[currentsection]
  \end{frame}
}

%---------------------------------------------------
%---------------- Document -------------------------
%---------------------------------------------------

\begin{document}

\frame{\titlepage}

%-----------------------------------------------------
\begin{frame}
\frametitle{Table of Contents}
\tableofcontents
\end{frame}
%----------------------------------------------------

\section{Introduction}

%---------------------------------------------------
\begin{frame}
\frametitle{Hidden and hard-to-reach populations}

\begin{itemize}
    \item No sampling frame exists: size and boundaries of the population are unknown.
    \item Privacy concerns: stigmatized or illegal behavior. 
    \item Fear of exposition or prosecution complicates the enumeration and learning about these populations.
    \item High logistic cost when the occurrence frequency is low.
    \item Examples: Heavy drug users, sex workers, homeless people, and men who have sex
    with men. 
\end{itemize}

\end{frame}

\begin{frame}
    
    \frametitle{Existing sampling methods}

    \begin{itemize}
        \item {\bf Snowball} \cite{goodman1961}
    \end{itemize}

    From starting individuals, each subject provides a list of names of known
    individuals from the target population. The researcher invites this person
    to participate, who can agree or deny it. 

    \Space

    \begin{itemize}
      \item {\bf Key informant} \cite{deaux-callaghan1985}
    \end{itemize}

    Expert respondents are selected to answer about others' behavior. For
    instance, social workers, drug abuse counselors, official, etc. 

    \Space

    \begin{itemize}
      \item {\bf Targeted} \cite{watters-biernacki1989}
    \end{itemize}

   Field researchers build an ethnographic mapping of a target population, and
   recruit a number of individuals at sites identified by this map.

\end{frame}

\begin{frame}

  \frametitle{Problems with snowball sampling}

  \begin{itemize}
    \item Inferences about the individuals depend on the initial sample. 
    \item Bias towards individuals who are more cooperative and agree to
    participate. 
    \item Bias because of masking, that is, protecting friends by not
    referring them. 
    \item Individuals with more links may be oversampled.  
  \end{itemize}
  
\end{frame}

\begin{frame}
  
  \frametitle{Respondent-driven sampling}

  \begin{enumerate}
    \item The researchers select a handful of individuals from a target
    population who serve as {\em seeds}.
    \item Each participant receives a fixed number of {\em recruitment coupons} and invite
    members of their own social network to participate in exchange of a
    reward.   
    \item The sampling is without replacement. 
    \item If the individual accepts to participate, they answer a
    questionnaire and inform the network degree. One important point is that
    the recruiter doesn't say the name of the other members, reducing the
    mask effect. 
  \end{enumerate}

\end{frame}

\begin{frame}
  
  \frametitle{System of incentives}
    
  Two different sources of theoretical incentive (dual incentive system):

  \Space

  \begin{itemize}
    \item {\bf Individual-sanction based control:} reward for participating
    in the research. 
    \item {\bf Group-mediated social control:} reward for recruiting peers.
    When social approval is important, it's more efficient and cheaper.
    Symbolic incentive is also important. 
  \end{itemize}

\end{frame}

%------------------------------------------------------

\section{Mathematical formulation}

\begin{frame}
  
  \frametitle{Modelo formal}

  The RDS can be seen mathematically in two different approaches. 

  \Space

  \begin{itemize}
    \item {\bf Stochastic process} \cite{heckathorn1997}
  \end{itemize}

  Each recruiter's social characteristics affect the characteristics of the
  recruits. There are a limited number of states that subjects can assume and
  the recruits are function of the recruiter characteristics.

  \Space

  \begin{itemize}
    \item {\bf Graphical structure} \cite{crawford2016}
  \end{itemize}

  A hidden population is an undirected graph, and we observe it partially in
  the {\em recruitment graph}, as also the coupon matrix and recruitment
  times. The unobserved graph is treated as {\em missing data} what can be
  interpreted as as Exponential Random Graph Model.  

\end{frame}

\begin{frame}
\frametitle{Markov chain model}



\end{frame}

\begin{frame}
\frametitle{Network model}  

\end{frame}

\section{Examples and usages}

\section{Applications with real data}

\begin{frame}[t, allowframebreaks]
   \frametitle{References}
   \bibliographystyle{apalike}
   \bibliography{biblio}
 \end{frame}

\end{document}