\documentclass{beamer}

%------------------------------------
%------------Libraries---------------
%------------------------------------

\usepackage[utf8]{inputenc}
\usepackage{xpatch}
\usepackage{xcolor}
\usepackage{url, hyperref}
\usepackage[authoryear,round]{natbib}

\usepackage{amsmath, amsthm, amssymb}

%------------------------------------
%----------Configurations------------
%------------------------------------

\usetheme{Madrid}
\usecolortheme{default}
\useinnertheme{circles}

\definecolor{FirstColor}{rgb}{0.0157,0.2392,0.4902}
\definecolor{SecondColor}{rgb}{0.0157, 0.549, 0.8}

\setbeamertemplate{itemize items}[triangle]

\setbeamercolor*{palette primary}{bg=FirstColor, fg=white}
\setbeamercolor*{palette secondary}{bg=SecondColor, fg=white}
\setbeamercolor*{palette tertiary}{bg=white, fg=FirstColor}
\setbeamercolor*{palette quaternary}{bg=FirstColor,fg=white}
\setbeamercolor{structure}{fg=FirstColor}
\setbeamercolor{section in toc}{fg=FirstColor}

\hypersetup{colorlinks=true,citecolor=blue, urlcolor = cien, linkcolor=blue}

%--- space between items 

\xpatchcmd{\itemize}
  {\def\makelabel}
  {\ifnum\@itemdepth=1\relax
     \setlength\itemsep{3ex}% separation for first level
   \else
     \ifnum\@itemdepth=2\relax
       \setlength\itemsep{2ex}% separation for second level
     \else
       \ifnum\@itemdepth=3\relax
         \setlength\itemsep{0.5ex}% separation for third level
   \fi\fi\fi\def\makelabel
  }
{}{}

%---------------------------------------------------
%----------------Front page-------------------------
%---------------------------------------------------

\title[Respondent driven-sampling]
{Respondent driven-sampling}
\subtitle{Procedure to sample from hidden or hard-to-reach populations}
\author[Lucas Moschen]
{Lucas Moschen}
\institute[EMAp/FGV]
{
  School of Applied Mathematics\\
  Fundação Getulio Vargas
}
\date[\today]
{\today}

\titlegraphic{
    \vspace*{1.2cm}
    \hspace*{9.5cm}
    \includegraphics[width=.2\textwidth]{images/logo-emap.png}
}
%--------------------------------------------------

\AtBeginSection[]
{
  \begin{frame}
    \frametitle{Table of Contents}
    \tableofcontents[currentsection]
  \end{frame}
}
%----------------------------------------------------

%alert and alertblock -> red
%block -> blue
%examples -> green 

\begin{document}

\frame{\titlepage}

%-----------------------------------------------------
\begin{frame}
\frametitle{Table of Contents}
\tableofcontents
\end{frame}
%----------------------------------------------------

\section{Introduction}

%---------------------------------------------------
\begin{frame}
\frametitle{Hidden and hard-to-reach populations}

\begin{itemize}
    \item No sampling frame exists: size and boundaries of the population are unknown.
    \item Privacy concerns: stigmatized or illegal behavior. 
    \item Fear of exposition or prosecution complicates the enumeration and learning about these populations.
    \item High logistic cost when the occurrence frequency is low.
    \item Examples: Heavy drug users, sex workers, homeless people, and men who have sex
    with men. 
\end{itemize}

\end{frame}

\begin{frame}
    
    \frametitle{Existing sampling methods}

    \begin{itemize}
        \item Snowball (\cite{goodman1961})
    \end{itemize}

    From starting individuals, each subject provides a list of names of known
    individuals from the target population. The researcher invites this person
    to participate, who can agree or deny it. 

    \vspace{2ex}

    \begin{itemize}
      \item Key informant (\cite{deaux-callaghan1985})
    \end{itemize}

    Expert respondents are selected to answer about others' behavior. For
    instance, social workers, drug abuse counselors, official, etc. 

    \vspace{2ex}

    \begin{itemize}
      \item Targeted (\cite{watters-biernacki1989})
    \end{itemize}

   Field researchers build an ethnographic mapping of a target population, and
   recruit a number of individuals at sites identified by this map.

\end{frame}

\begin{frame}

  \frametitle{Problems with snowball sampling}

  \begin{itemize}
    \item Inferences about the individuals depend on the initial sample. 
    \item Bias towards individuals who are more cooperative and agree to
    participate. 
    \item Bias because of masking, that is, protecting friends by not
    referring them. 
    \item Individuals with more links may be oversampled.  
  \end{itemize}
  
\end{frame}

\begin{frame}
  
  \frametitle{Respondent-driven sampling}

\end{frame}

%------------------------------------------------------

\section{Mathematical formulation}

\begin{frame}
\frametitle{Markov chain}

\end{frame}

\begin{frame}
\frametitle{Network model}  

\end{frame}

\section{Examples and usages}

\section{Applications with real data}

\begin{frame}[t, allowframebreaks]
  \frametitle{References}
  \bibliographystyle{apalike}
  \bibliography{biblio}
\end{frame}

\end{document}