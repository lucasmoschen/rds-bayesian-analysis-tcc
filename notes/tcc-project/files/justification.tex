There are two justifications for the importance of this work. First, hidden
populations are often omitted from national representative surveys since they
do not have fixed addresses or fear prosecution. However, the individuals can
have a greater risk of drug abuse or having sexually transmitted infections.
This combination creates an environment of aid absence from the government to
these people. The second reason is mathematical. This topic has lots of gaps
in Statistics that deserve attention. The correct sampling probabilities for
the recruited members under RDS are hard to obtain since not all links and
nodes are observed, constituting missing data \cite{crawford2016}. In this fertile area,
regression approaches to prevalence estimation taking the network structure
can be built \cite{bastos2012binary} and are still in development.