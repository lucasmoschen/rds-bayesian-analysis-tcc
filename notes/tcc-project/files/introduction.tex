\begin{citacao}
O texto deve ser constituído de uma parte introdutória, na qual devem ser
expostos o tema do projeto, o problema a ser abordado, a(s) hipótese(s),
quando couber(em), bem como o(s) objetivo(s) a ser(em) atingido(s) e a(s)
justificativa(s). É necessário que sejam indicados o referencial teórico que
o embasa, a metodologia a ser utilizada, assim como os recursos e o cronograma
necessários à sua consecução.
\end{citacao}

The proposal of this work is to study the survey method Respondent-Driven
Sampling, a chain-referral method with the objetive of sample from
hard-to-reach populations, when it is necessary to estimate the prevalence of
some characteristic from this population. It is also regarded the imperfection
of the tests of this characteristic, so sensibility and sensitivity are
accounted in the model.  

Hidden or hard-to-reach populations have two main features: no sampling frame
exists, so size and boundaries of the population are unknown; and there are
heavy privacy concerns because the subjects are stigmatized or have illegal
behavior \cite{heckathorn1997}. Fear of exposition of prosecution complicates
the enumeration of the populations and the learning about them. Moreover, if
the occurrence frequency of the characteristic is low, there is high logistic
cost involved. Some examples are heavy drug users, sex workers, homeless
people, and men who have sex with men. 

Some methods were developed to reach these populations, such as, for
example, snowball method \cite{goodman1961}, key important method
\cite{deaux-callaghan1985}, and targeted method \cite{watters-biernacki1989}.
\citeauthor{heckathorn1997} introduced the Respondent-Driven Sampling (RDS) to
fill some gaps he depicted in his work. In his proposed method, the researches
select a handful of individuals from the target population and give them
coupons to recruit others from the population. The individuals receive a
reward for being recruited and for each recruitment, which creates a dual
incentive system. Several papers after \citeyear{heckathorn1997} were written.

After sampling individuals from the target population, a questionnaire is
conducted, and in this work takes questions with binary outcome. 

\lucas{Falar um pouco de como esse estudo vai ser feito: estatística bayesiana}

\lucas{Falar um pouco do objetivo deste trabalho.}


\section{Respondent-driven sampling}


\section{Regression with binary outcome with imperfect tests}


\section{Bayesian statistics}

Respondant-Driven Sampling (RDS) é um procedimento utilizado para amostrar
populações de difícil acesso, como exemplo a população de usuários de drogas
pesadas e profissionais do sexo. Ele funciona de forma similar a um processo de
ramificação em formato de rede, em que os participantes em cada estágio recrutam,
em sua própria sub-rede, os próximos participantes e o primeiro estágio é chamado de
semente.

Esse método pode ser utilizado em forma de pesquisa a fim de estimar a
prevalência de alguma característica, isto é, o número total de indivíduos que
possuem determinada característica. Nessa pesquisa, cada participante responde uma
série de perguntas relacionadas ao objeto de estudo e outras covariáveis. Vamos
considerar neste trabalho que o desfecho de interesse é uma variável binária e sujeita
a erro de medição, isto é, não é possível ter certeza sobre a veracidade da resposta
dada. Usamos os conceitos de sensitividade e especificidade para lidar com
isso.

Todavia, em vista de nosso desconhecimento sobre a natureza em si, se faz
necessário modelar a incerteza dessas variáveis e, para tanto, a Estatística bayesiana é
a área de estudo indicada. A ideia, portanto, é propagar a incerteza sobre a resposta
dos participantes pela rede de contatos.

Por fim, pretende-se aplicar esse framework de forma eficiente, em particular,
comparando os algoritmos de Markov chain Monte Carlo e Aproximação de Laplace
aninhada (INLA) e programando-os com ajuda de alguma linguagem de programação
como R, Stan ou Python.