\documentclass{beamer}

%------------------------------------
%------------Libraries---------------
%------------------------------------

\usepackage[utf8]{inputenc}
\usepackage{xpatch}
\usepackage{ragged2e}
\usepackage{xcolor}
\usepackage{url, hyperref}

\usepackage[percent]{overpic}

\usepackage{amsmath, amsthm, amssymb, amsfonts} 

%------------------------------------
%----------Configurations------------
%------------------------------------

\usetheme{Madrid}
\usecolortheme{default}
\useinnertheme{circles}

\definecolor{FirstColor}{rgb}{0.0157,0.2392,0.4902}
\definecolor{SecondColor}{rgb}{0.0157, 0.549, 0.8}

\setbeamertemplate{itemize items}[triangle]

\setbeamercolor*{palette primary}{bg=FirstColor, fg=white}
\setbeamercolor*{palette secondary}{bg=SecondColor, fg=white}
\setbeamercolor*{palette tertiary}{bg=white, fg=FirstColor}
\setbeamercolor*{palette quaternary}{bg=FirstColor,fg=white}
\setbeamercolor{structure}{fg=FirstColor}
\setbeamercolor{section in toc}{fg=FirstColor}

\hypersetup{colorlinks=true,citecolor=blue, urlcolor = cien, linkcolor=blue}

\apptocmd{\frame}{}{\justifying}{}

%---------------------------------------------------
%------------------Itemize--------------------------
%---------------------------------------------------

\makeatletter
\newcommand{\my@beamer@setsep}{%
\ifnum\@itemdepth=1\relax
     \setlength\itemsep{\my@beamer@itemsepi}% separation for first level
   \else
     \ifnum\@itemdepth=2\relax
       \setlength\itemsep{\my@beamer@itemsepii}% separation for second level
     \else
       \ifnum\@itemdepth=3\relax
         \setlength\itemsep{\my@beamer@itemsepiii}% separation for third level
   \fi\fi\fi}
\newlength{\my@beamer@itemsepi}\setlength{\my@beamer@itemsepi}{3ex}
\newlength{\my@beamer@itemsepii}\setlength{\my@beamer@itemsepii}{1.5ex}
\newlength{\my@beamer@itemsepiii}\setlength{\my@beamer@itemsepiii}{1.5ex}
\newcommand\setlistsep[3]{%
    \setlength{\my@beamer@itemsepi}{#1}%
    \setlength{\my@beamer@itemsepii}{#2}%
    \setlength{\my@beamer@itemsepiii}{#3}%
}
\xpatchcmd{\itemize}
  {\def\makelabel}
  {\my@beamer@setsep\def\makelabel}
 {}
 {}

\xpatchcmd{\beamer@enum@}
  {\def\makelabel}
  {\my@beamer@setsep\def\makelabel}
 {}
 {}
\makeatother

%---------------------------------------------------
%-----------------Definitions-----------------------
%---------------------------------------------------

\newcommand{\Space}{\vspace{3ex}}

%---------------------------------------------------
%----------------Front page-------------------------
%---------------------------------------------------

\title[Respondent driven-sampling]
{Respondent driven-sampling}
\subtitle{Bayesian analysis of respondent-driven survey with outcome uncertainty}
\author[Lucas Moschen]
{Lucas Moschen}
\institute[EMAp/FGV]
{
  School of Applied Mathematics\\
  Fundação Getulio Vargas
}
\date[\today]
{\today}

\titlegraphic{
    \vspace*{1.2cm}
    \hspace*{9.5cm}
    \includegraphics[width=.2\textwidth]{../rds-presentation/images/logo-emap.png}
}
%--------------------------------------------------

\AtBeginSection[]
{
  \begin{frame}
    \frametitle{Table of Contents}
    \tableofcontents[currentsection]
  \end{frame}
}

%---------------------------------------------------
%---------------- Document -------------------------
%---------------------------------------------------

\begin{document}

\frame{\titlepage}

%-----------------------------------------------------
\begin{frame}
\frametitle{Table of Contents}
\tableofcontents
\end{frame}
%----------------------------------------------------

\section{Introduction}

%---------------------------------------------------

\begin{frame}
\frametitle{Hidden and hard-to-reach populations}

\begin{itemize}
    \justifying
    \item No sampling frame exists: size and boundaries of the population are unknown.
    \item Privacy concerns: stigmatized or illegal behavior. 
    \item Fear of exposition or prosecution complicates the enumeration and learning about these populations.
    \item High logistic cost when the occurrence frequency is low.
    \item Examples: Heavy drug users, sex workers, homeless people, and men who have sex
    with men. 
\end{itemize}

\end{frame}

\begin{frame}
\frametitle{Respondent-driven sampling}

  \begin{enumerate}
    \justifying
    \item The researchers select a handful of individuals from a target
    population who serve as {\em seeds}.
    \item Each participant receives a fixed number of {\em recruitment coupons} and invite
    members of their own social network to participate in exchange of a
    reward.   
    \item The sampling is without replacement. 
    \item If the individual accepts to participate, they answer a
    questionnaire and inform the network degree. One important point is that
    the recruiter doesn't say the name of the other members, reducing the
    mask effect. 

  \end{enumerate}

\end{frame}

\begin{frame}
  
  \frametitle{Formal model}

  The RDS can be seen mathematically in two different approaches. 

  \Space

  \begin{itemize}
    \item {\bf Stochastic process} \cite{heckathorn1997}
  \end{itemize}

  Each recruiter's social characteristics affect the characteristics of the
  recruits. There are a limited number of states that subjects can assume and
  the recruits are function of the recruiter characteristics.

  \Space

  \begin{itemize}
    \item {\bf Graphical structure} \cite{crawford2016}
  \end{itemize}

  A hidden population is an undirected graph, and we observe it partially in
  the {\em recruitment graph}, as also the coupon matrix and recruitment
  times. The unobserved graph is treated as {\em missing data} what can be
  interpreted as as Exponential Random Graph Model.  

\end{frame}

\begin{frame}{Prevalence estimation with imperfect tests}

  \begin{itemize}
    \item Prevalence ($\theta$) = Proportion of a disease (or condition) at time $t$;
    
    \item {\em Specificity} $(\gamma_s)$ and {\em Sensitivity} ($\gamma_e$) are (jointly) measures of
    the disease diagnose accuracy;

    \item If a sample $\{y_1, ..., y_n\}$ is observed, its mean is the maximum
    likelihood estimator, but it supposes a perfect test; 

    \item If $\pi$ is the probability of a positive test, 
    $$
    \pi = \theta\gamma_s + (1-\theta)(1-\gamma_e),
    $$
    and we can study $\theta$ regarding $\gamma_s, \gamma_e$. 

    \item Observe also that for a network the independence of the sample is
    also not valid. Other estimators are better. 

  \end{itemize}

\end{frame}

\begin{frame}{Bayesian statistics}

  \begin{itemize}
    \item Interpretation based on an individual's degree of belief in a statement;

    \Space

    \item Bayes' formula relates the probability of a parameter after
    observing new data with evidente and previous information about it;

    \Space

    \item It allows the quantification of uncertainty in a straightforward
    way: the process do not need to be random.
  \end{itemize}

\end{frame}

%----------------------------------------------------

\section{Justification}

\begin{frame}{Justification}

  \begin{itemize}
    \item Hidden populations are often omitted from national representative
    surveys and have higher risk of drug abuse or sexually transmitted
    infections. 

    \Space
    
    \item The topic has a lot of gaps in Statistics and regression approaches
    to prevalence estimation taking the network structure can be built \cite{bastos2012binary}.
  \end{itemize}
\end{frame}

%---------------------------------------------------

%----------------------------------------------------

\section{Objectives}

\begin{frame}{Main objective}

  The objective of this work is to analyze the network structure of RDS as a
stochastic object, along with the sensibility and sensitivity. We also intend
to apply this framework efficiently, comparing Monte Carlo algorithms and
Laplace approximations.
  
\end{frame}

\begin{frame}{Specific}

  \begin{enumerate}
    \item Bibliography review; 

    \item Problem description in mathematical terms and uncertainty propagation; 

    \item Bayesian methods and prior calibration; 

    \item Joint prior distribution for sensitivity and specificity;

    \item Efficient implementation using statistical packages, as {\em
    rstanarm} and {\em INLA}. Comparison between MCMC and Laplace
    approximation; 

    \item Analysis of RDS epidemiological studies.
  \end{enumerate}
  
\end{frame}

%---------------------------------------------------

%----------------------------------------------------

\section{Methodology}

%---------------------------------------------------

\begin{frame}{Methodology}

{\bf Document research}

The theoretical foundation will be through papers in the topics indicated in
the introduction, RDS, bayesian statistics, and prevalence estimation through
regression. 

\Space

{\bf Technical resources}

All the necessary programming will be done in the programming languages
\textit{Python} and \textit{R}. 

\Space

{\bf Formal study}

Two subjects from the PhD in Mathematical Modelling at EMAp will be taken:
Bayesian Statistics and Network Science.

\end{frame}

%----------------------------------------------------

\section{Preliminary results}

%---------------------------------------------------

\begin{frame}{RDS studies: refugees and activists in Syria}

\begin{figure}
  \centering
  \begin{overpic}[width=\textwidth]{../../images/graph-rds-harvard.png}
    \put (50,20) {Network generated by a RDS study}
    \put (50,15) {with refugees and activists in Syria.}
    \put (50,10) {We see part of the graph.}
   \end{overpic}
\end{figure}
  
\end{frame}

\begin{frame}{Regression}
  \begin{enumerate}
    \item Adicionar o modelo 1 e comentar sobre a beta bivariada. 
  \end{enumerate}

\end{frame}

%----------------------------------------------------

\section{Schedule}

\begin{frame}{Schedule}
  
\end{frame}

%---------------------------------------------------

\begin{frame}[t, allowframebreaks]
   \frametitle{References}
   \bibliographystyle{apalike}
   \bibliography{../tcc-project/biblio}
 \end{frame}

\end{document}