\chapter{Introduction}

Hidden or hard-to-reach populations have two key features: no sampling frame
exists, and the individuals have privacy concerns about participating in
surveys. The former occurs because the population's size and boundaries are
unknown. Furthermore, the subjects suffer from stigmatization or have illegal
behaviour \cite{heckathorn1997}, which complicates the learning about them. Moreover, if the
occurrence frequency of the condition is low, there are high logistic costs
involved. Some examples are heavy drug users, sex workers, homeless people,
and men who have sex with men.

Research has been carried out with the development of some methods to reach these populations, such as, for example, snowball sampling \cite{goodman1961}, key
important sampling \cite{deaux-callaghan1985}, 
and targeted sampling \cite{watters-biernacki1989}. \textcite{heckathorn1997} introduced the Respondent-Driven Sampling (RDS) to fill some gaps from other methods he depicted in his work. In his proposed approach, the researchers select a handful of individuals from the target population and give them coupons to recruit their peers. The individuals receive a reward for being recruited and for recruiting, which creates a dual incentive system. After \textcite{heckathorn1997}, several papers studied this topic more deeply. 

After sampling from the target population, a questionnaire or a disease test is conducted. This work considers binary outcomes, such as smoking status or testing for HIV infections. When considering a disease, an essential quantity for epidemiologists is the proportion of infected people, called the prevalence. The estimation of this parameter can be crucial for public decision making. 

However, the diagnoses are subject to measurement error, and their accuracy is
a vital step \cite{reitsma2005bivariate}. One common way to do this is jointly measuring {\em
sensitivity} and {\em specificity}. Sensitivity measures the ability of the
test to detect the condition, while specificity the absence of it. With that
in mind, we consider both misclassification and RDS structure to infer the
prevalence. 

Because of our lack of knowledge about nature itself, it is necessary to model the uncertainty of this process, and Bayesian Statistics is the recommended area of study. In this paradigm, the parameters are random variables, and new data update the beliefs about them. The idea is to propagate outcome uncertainty through the network of contacts. We develop a hierarchical model to represent the structure of the process. 

This work proposes to study the survey method RDS, a chain-referral method to interview hard-to-reach populations when necessary to estimate the prevalence of some binary condition from this population. The modelling also accounts for sensibility and sensitivity since the imperfection of the detection tests. We also apply these methods using the Hamiltonian Monte Carlo method, with an analysis of its results.