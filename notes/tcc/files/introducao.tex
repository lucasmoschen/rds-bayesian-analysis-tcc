\chapter{Introduction}

Hidden or hard-to-reach populations have two key features: no sampling frame
exists, and the individuals have privacy concerns about participating in
surveys. The former occurs because the population's size and boundaries are
unknown. Furthermore, the subjects suffer from stigmatization or engage in illegal
behaviour \cite{heckathorn1997}, which complicates learning about them.
Moreover, if the frequency of the condition of interest is low, there are high logistic
costs
involved. Some examples are heavy drug users, sex workers, homeless people,
and men who have sex with men.

Methods to reach these populations, such as, for example, snowball sampling \cite{goodman1961}, key
important sampling \cite{deaux-callaghan1985}, 
and targeted sampling \cite{watters-biernacki1989} have been developed. \textcite{heckathorn1997} introduced the Respondent-Driven Sampling (RDS) to fill some gaps from other strategies he depicted in his work. In his proposed approach, the researchers select a handful of individuals from the target population and give them coupons to recruit their peers. The individuals receive a reward for being recruited and for recruiting, which creates a dual incentive system. After \textcite{heckathorn1997}, several papers studied this topic more deeply. 

After sampling from the target population, a questionnaire or a disease test
is conducted. When considering a disease, an essential quantity
for epidemiologists is the proportion of infected people, called the
prevalence. The estimation of this quantity can be crucial for public
decision making. However, the diagnoses are subject to measurement error, and considering
their accuracy is
a vital step \cite{reitsma2005bivariate}. One common way to do this is jointly measuring {\em
sensitivity} and {\em specificity}. Sensitivity measures the ability of the
test to detect the condition, whereas specificity refers to the test's
capacity to verify its absence. These quantities are often negatively
correlated, and a higher sensitivity of a screening test reduces the
specificity and vice versa.

Furthermore, other variables accessed in the survey can be possible risk
factors for the disease and need better understanding within the target
population. From correlational studies, one can analyze causal dependencies. 
The literature of
regression analysis in RDS samples is not established yet \cite[p.
15]{avery2021statistical} and conceptual questions are yet to be addressed. 
We develop a hierarchical model to represent the structure of the process. 

Because of our lack of knowledge about the complex interactions in Nature, it
is necessary to model the uncertainty of the process under study. The models
we develop to represent these interactions are approximations subjected to
error. It leads to the famous quote: "all models are wrong, but some are
useful." In finite sample experiments, the model parameters should reflect our
state-of-art knowledge about them. The Bayesian Statistics paradigm handles
uncertainty quantification defining the unknown quantities as random
variables, and new data update the beliefs about these quantities.  

This work proposes to study the survey method RDS, a chain-referral method to
interview hard-to-reach populations when necessary to estimate the prevalence
of some binary condition from this population. The modelling strategy follows
a hierarchical model with regressor variables. The modelling also accounts for
sensibility and sensitivity given the imperfection of the detection tests. We
apply these methods using the Hamiltonian Monte Carlo method in Simulated and
real data.

The dissertation is organized as follows: Chapter \ref{ch:theoretical-background}
describes the theoretical foundations. Chapter \ref{ch:modelling} discusses
each block of the model and validades its properties in simulated data. Model
extensions are also discussed in the end of the chapter.
Chapter \ref{ch:real_data_applications} describes two datasets and utilize the
model on them. At last, Chapter \ref{ch:conclusions} concludes the main aspects of
this work.  