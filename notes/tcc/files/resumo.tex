\setlength{\absparsep}{18pt} 

\begin{resumo}[Abstract]
 \begin{otherlanguage*}{english}
    Hard-to-reach populations are difficult to access for researchers or refuse to enrol in public health surveys, making enumeration and sampling challenges. Respondent-driven sampling (RDS) is a chain-referral technique used to recruit individuals from hard-to-reach populations. The survey encourages the participants to recruit their peers, giving incentives to each recruitment and for participation. Since there is no enumeration of the subjects, RDS is a non-probabilistic sampling strategy. Moreover, the graphical structure of RDS suffers from missing data, and several assumptions about the recruitment process are necessary. 
    
    After having the sampled individuals, understanding their characteristics
    is a focus in epidemiology, given that these are usually high-risk
    populations to some diseases. Therefore, estimating the disease
    prevalence, the proportion of infected individuals, and the dependence
    among other observed variables is a critical step for public decision
    making. Diagnostic tests for disease identification are subject to
    misclassification, and incorporating their accuracy corrects biases in the
    prevalence estimation problem. This work proposes the use of regression
    techniques for prevalence estimation in respondent-driven samples. We use
    conditionally autoregressive models to represent correlation among the
    individuals induced by recruitment.
    
    In modern statistics, understanding situations with unknown information
    and quantifying them plays a significant role. We use Bayesian inference
    for uncertainty quantification for our models. In the Bayesian paradigm,
    probability distributions for quantities of interest represent the belief
    about them. We discuss different prior specification approaches for the
    parameters and examine uncertainty about the graph structure using a
    graphical model of RDS. To perform sampling from the parameter
    distribution, we used the Hamiltonian Monte Carlo sampler. Diagnostics of
    this method helped to improve our model programming. Verification of the
    model through simulation and external datasets showed robust results, and
    we propose model extensions for the limitations of this work.
 \end{otherlanguage*}=

 Keywords: respondent-driven sampling, regression analysis, Bayesian
 inference, prevalence estimation, misclassification, sensitivity, specificity
\end{resumo}

\begin{resumo}[Resumo]

    Populações de difícil acesso são difíceis para pesquisadores se aproximarem
    ou se recusam a se inscrever em pesquisas de saúde pública, tornando a
    enumeração e amostragem desafios. O Respondent-driven sampling
    (RDS) é uma técnica de referência em cadeia usada para recrutar indivíduos
    de populações difíceis de alcançar. A pesquisa incentiva os participantes
    a recrutarem seus pares, dando incentivos a cada recrutamento e à
    participação. Como não há enumeração dos sujeitos, o RDS é uma estratégia
    de amostragem não probabilística. Além disso, a estrutura gráfica do RDS
    sofre com a falta de dados e várias suposições sobre o processo de
    recrutamento são necessárias. 

    Após a obtenção dos indivíduos amostrados, o entendimento de suas
    características é foco da epidemiologia, visto que se trata de
    populações geralmente de alto risco para algumas doenças. Portanto,
    estimar a prevalência da doença, a proporção de indivíduos infectados e a
    dependência a outras variáveis observadas é uma etapa crítica para a
    tomada de decisão pública. Os testes de diagnóstico para identificação de
    doenças estão sujeitos a erros de classificação e incorporar suas
    acurácias corrige vieses no problema de estimação de prevalência. Este
    trabalho propõe o uso de técnicas de regressão para estimativa de
    prevalência em Respondent-driven sampling. Usamos modelos
    condicionalmente autorregressivos para representar a correlação entre os
    indivíduos induzida pelo recrutamento. 
    
    Na estatística moderna, entender situações com informações desconhecidas e
    quantificá-las desempenha um papel significativo. Usamos inferência
    bayesiana para quantificação de incerteza para nossos modelos. No
    paradigma bayesiano, as distribuições de probabilidade para quantidades de
    interesse representam a crença sobre elas. Discutimos diferentes
    abordagens de especificação anterior para os parâmetros e examinamos a
    incerteza sobre a estrutura do grafo usando um modelo gráfico de RDS.
    Para realizar a amostragem da distribuição dos parâmetros, usamos o
    amostrador Hamiltonian Monte Carlo. Os diagnósticos deste método ajudaram a
    melhorar a programação do nosso modelo. A verificação do modelo por meio
    de simulação e conjuntos de dados externos mostrou resultados robustos, e
    propomos extensões do modelo para as limitações deste trabalho. 
    
    Palavras-chave: respondent-driven sampling, análise de regressão,
    inferência bayesiana, estimação de prevalência, classificação errada,
    sensibilidade, especificidade
\end{resumo}