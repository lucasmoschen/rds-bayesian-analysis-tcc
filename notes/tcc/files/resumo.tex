\setlength{\absparsep}{18pt} 

\begin{resumo}[Abstract]
 \begin{otherlanguage*}{english}
    Respondent-driven sampling (RDS) is a chain-referral technique used to enrol individuals from hard-to-reach populations, which are hard to access and enumerate. The survey financially encourages the participants to recruit their peers. Since there is no enumeration of the subjects, RDS is a non-probabilistic sampling strategy. Moreover, the graphical structure of RDS suffers from missing data. Statistical methods of the literature are constantly being developed by academic research. However, several assumptions about the recruitment process are necessary. 

    After having the sampled individuals, understanding its characteristics is a focus in epidemiology, given that these are usually high-risk populations to some diseases. Therefore, estimating the disease prevalence, the proportion of infected individuals, and the dependence among other observations is a critical step for public decision making. Diagnostic tests for disease identification are subject to misclassification, and incorporating their accuracy corrects biases in the prevalence estimation problem. This work proposes the use of regression techniques for prevalence estimation in respondent-driven samples. We use conditionally autoregressive models to represent spatial correlation among the individuals.

    For uncertainty quantification, we employ Bayesian Inference in our models with prior specification analysis. The parameter distribution samples were obtained through Hamiltonian Monte Carlo (HMC) sampler. We analyse the diagnostics of this method in our model. We examined uncertainty about the graph structure using a graphical model of RDS. Verification of the model through simulation and external datasets was performed. We achieved interesting results with several more questions about the model's structure and some extensions.
 \end{otherlanguage*}

 Keywords: respondent-driven sampling, regression analysis, Bayesian
 inference, prevalence estimation, misclassification, sensitivity, specificity
\end{resumo}

\begin{resumo}[Resumo]
    O Respondent-driven sampling (RDS) é uma técnica de referência em cadeia usada para investigar indivíduos de populações de difícil acesso, que são difíceis de enumerar. A pesquisa incentiva financeiramente os participantes a recrutarem seus pares. Como não há enumeração dos sujeitos, o RDS é uma estratégia de amostragem não probabilística. Além disso, a estrutura gráfica do RDS sofre com a falta de dados. Os métodos estatísticos da literatura estão constantemente sendo desenvolvidos por pesquisas acadêmicas. No entanto, várias suposições sobre o processo de recrutamento são necessárias.

    Após a obtenção dos indivíduos amostrados, o entendimento de suas características passa a ser foco da epidemiologia, visto que se trata de populações geralmente de alto risco para algumas doenças. Portanto, estimar a prevalência da doença, a proporção de indivíduos infectados, e a dependência entre outras observações, é uma etapa crítica para a tomada de decisão pública. Os testes de diagnóstico para identificação de doenças estão sujeitos a erros de classificação e a incorporação de sua precisão corrige vieses no problema de estimativa de prevalência. Este trabalho propõe o uso de técnicas de regressão para estimativa de prevalência em amostras RDS. Usamos modelos condicionalmente autorregressivos para representar a correlação espacial entre os indivíduos.

    Para a quantificação da incerteza, empregamos inferência bayesiana em
    nossos modelos com análise de especificação de prioris. As amostras de
    distribuição dos parâmetros foram obtidas por meio do amostrador
    Hamiltonian Monte Carlo (HMC). Analisamos o diagnóstico deste método
    em nosso modelo. Examinamos a incerteza sobre a estrutura do gráfico
    usando um modelo gráfico de RDS. A verificação do modelo por meio de
    simulação e conjuntos de dados externos foram realizados. Obtivemos
    resultados interessantes com várias outras questões sobre a estrutura do
    modelo e algumas extensões.
    
    Palavras-chave: respondent-driven sampling, análise de regressão,
    inferência bayesiana, estimação de prevalência, classificação errada,
    sensibilidade, especificidade
\end{resumo}