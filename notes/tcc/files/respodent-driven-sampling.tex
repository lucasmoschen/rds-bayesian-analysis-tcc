\chapter{Respondent-driven sampling}

Respondent-driven sampling (RDS) is commonly used to survey hidden or hard-to-reach populations when
no sampling frame exists \cite[]{heckathorn1997}, which means there is no
enumeration of the population, since size and boundaries are unknown. In this approach, the
researchers select some individuals, called {\em seeds} from the target
population, and give them a fixed amount of {\em recruitment coupons} to
recruit their peers. Each recipient of the coupons reclaims it in the study
site, is interviewed, and receives more coupons to continue the recruitment.
This process occurs until some criteria is reached. The sampling is without
replacement, so the participants cannot be recruited more than once. Moreover,
the respondents inform how many subjects from the population they know.

The subjects receive a reward for being interviewed and for each recruitment
of their peers which establishes a dual incentive system. The {\em primary incentive} is the
{\em individual-sanction-based control}, so there is a reward for
participating. The second one is the {\em group-mediated social control} that
influences the participants to induce others to comply to get the reward for the recruitment. When social approval is important, recruitment can be even
more efficient and cheaper, since material incentive can be converted into
symbolic by the individuals. In summary, accepting to be recruited will have a
material incentive for both and a symbolic incentive for the recruited, since
theirs peers also participated.

Let $G = (V,E)$ be an undirected graph representing the hidden population. The {\em recruitment graph} $G_R =
(V_R, E_R)$ represents the recruited individuals and the recruitment edges,
that is, $(i,j) \in E_R$ if, and only if, $i$ recruited $j$.
Given that each individual can be sampled only once, it is not possible to
observe the {\em recruitment-induced subgraph}, that is the induced subgraph
generated by $V_R$. Moreover, the {\em coupon matrix} $C$ defined by $C_{ij} =
1$ if the i$^{th}$ subject has at least one coupon before the j$^{th}$
recruitment event, is also observed with the recruitment times. Assuming an
exponential and independent distribution of the times, the likelihood can be
written explicitly, and the distribution interpreted as an exponential random graph
model \cite[]{crawford2016}.  

These models allowed several applications in social sciences, epidemiology,
and statistics, including hidden populations size estimation
\cite[]{crawford2018hidden}, regression \cite[]{bastos2012binary}, communicable
disease prevalence estimation \cite[]{albuquerque2009avaliaccao}, among others.
