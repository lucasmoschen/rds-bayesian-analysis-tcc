\chapter{Bivariate Beta distribution}
\label{appendix:bivariate-beta-distribution}

Let $U = (U_1, U_2, U_3, U_4) \sim
\operatorname{Dirichlet}(\boldsymbol{\alpha})$, where $\boldsymbol{\alpha} =
(\alpha_1, \alpha_2, \alpha_3, \alpha_4)$ with $\alpha_i > 0, i = 1,\dots,4$
and $U_4 = 1 - U_1 + U_2 + U_3$. The joint density of $U$ with respect to the
Lebesgue measure is given by
\begin{equation}
  f_U(u_1, u_2, u_3) = \frac{1}{B(\boldsymbol{\alpha})}u_1^{\alpha_1-1}u_2^{\alpha_2-1}u_3^{\alpha_3-1}(1-u_1-u_2-u_3)^{\alpha_4-1}, 
\end{equation}
when $u_i \in [0,1], i = 1,2,3$, $u_1 + u_2 + u_3 \le 1$, and $0$ otherwise.
The normalizing constant is, for $v \in \R^n$,
$$B(v) = \frac{\prod_{i=1}^n \Gamma(v_i)}{\Gamma\left(\sum_{i=1}^n v_i\right)}.$$ 

\begin{definition}
  Let 
  \begin{equation}
    X = U_1 + U_2 \text{ and } Y = U_1 + U_3.
  \end{equation} 
    The distribution of $(X,Y)$ is {\em Bivariate Beta} with parameters
    $\boldsymbol{\alpha}$. 
\end{definition}

\begin{proposition}
  The marginal distribution of $X$ is Beta with parameters $\alpha_1 +
  \alpha_2$ and $\alpha_3 + \alpha_4$. Similarly, the marginal distribution of
  $Y$ is Beta with parameters $\alpha_1 + \alpha_3$ and $\alpha_2 + \alpha_4$.
\end{proposition}

\begin{proof}
  First we derive the probability density of $(U_1, U_2)$ with respect to the
  Lebesgue measure. 
  \begin{equation}
    \label{eq:dist-u1-u2}
    \begin{split}
      f_{U_1, U_2}(u_1, u_2) &= \int_{-\infty}^{\infty} f_{U}(u_1,u_2,u_3) \, du_3 \\ 
      &= \frac{1}{B(\boldsymbol{\alpha})}\int_0^1 u_1^{\alpha_1-1}u_2^{\alpha_2-1}u_3^{\alpha_3-1}(1-u_1-u_2-u_3)^{\alpha_4-1} \, du_3 \\
      &= \frac{1}{B(\boldsymbol{\alpha})}u_1^{\alpha_1-1}u_2^{\alpha_2-1}\int_0^1 u_3^{\alpha_3-1}(1-u_1-u_2-u_3)^{\alpha_4-1} \, du_3.
    \end{split}
  \end{equation}
  Let $u_3 = (1 - u_1 - u_2)z$. Then,
  \begin{equation}
    \begin{split}
      f_{U_1, U_2}(u_1, u_2) &= \frac{1}{B(\boldsymbol{\alpha})}u_1^{\alpha_1-1}u_2^{\alpha_2-1}\int_0^1 (1-u_1-u_2)^{\alpha_3-1}z^{\alpha_3-1}(1-u_1-u_2)^{\alpha_4}(1-z)^{\alpha_4-1} \, dz. \\
      &= \frac{1}{B(\boldsymbol{\alpha})}u_1^{\alpha_1-1}u_2^{\alpha_2-1}(1-u_1-u_2)^{\alpha_3+\alpha_4-1}\int_0^1 z^{\alpha_3-1}(1-z)^{\alpha_4-1} \, dz. \\
      &= \frac{1}{B(\boldsymbol{\alpha})}u_1^{\alpha_1-1}u_2^{\alpha_2-1}(1-u_1-u_2)^{\alpha_3+\alpha_4-1}\frac{\Gamma(\alpha_3)\Gamma(\alpha_4)}{\Gamma(\alpha_3 + \alpha_4)} \\
      &= \frac{1}{B(\alpha_1, \alpha_2, \alpha_3+\alpha_4)}u_1^{\alpha_1-1}u_2^{\alpha_2-1}(1-u_1-u_2)^{\alpha_3+\alpha_4-1}.
    \end{split}
  \end{equation}

We conclude that
$$(U_1, U_2, 1-U_1-U_2) \sim
\operatorname{Dirichlet}(\alpha_1,\alpha_2,\alpha_3+\alpha_4).$$

Define 
$$
H(v) = \begin{bmatrix}
  1 & 0 \\ 1 & 1
\end{bmatrix}v, \text{ for } v \in \R^2.
$$

Then $(U_1, X) = H(U_1, U_2)$ and $H(\cdot)$ is bijective and differentiable function. By the Change of Variable Formula, 
\begin{equation}
  \begin{split}
    f_{U_1, X}(u_1, x) &= f({H^{-1}(u_1,x)})\bigg|\det\left[\frac{dH^{-1}(v)}{dv}\bigg|_{v=(u_1,x)}\right]\bigg| \\ 
    &= f(u_1, x - u_1) = \frac{1}{B(\alpha_1, \alpha_2, \alpha_3+\alpha_4)}u_1^{\alpha_1-1}(x-u_1)^{\alpha_2-1}(1-x)^{\alpha_3+\alpha_4-1}, 
  \end{split}
\end{equation}
where $(u_1, x)$ belongs to the triangle defined by the points (0,0),
(0,1), and (1,1). The distribution of $X$ for $x \in [0,1]$ is
\begin{equation}
  \begin{split}
    f_X(x) &= \frac{1}{B(\alpha_1, \alpha_2, \alpha_3+\alpha_4)}\int_{0}^{x} u_1^{\alpha_1-1}(x-u_1)^{\alpha_2-1}(1-x)^{\alpha_3+\alpha_4-1} \, du_1 \\
    &= \frac{1}{B(\alpha_1, \alpha_2, \alpha_3+\alpha_4)}(1-x)^{\alpha_3+\alpha_4-1} \int_{0}^{x} u_1^{\alpha_1-1}(x-u_1)^{\alpha_2-1} \, du_1. \\
    &= \frac{1}{B(\alpha_1, \alpha_2, \alpha_3+\alpha_4)}(1-x)^{\alpha_3+\alpha_4-1} \int_{0}^{x} x^{\alpha_1-1} \left(\frac{u_1}{x}\right)^{\alpha_1-1}x^{\alpha_2 - 1}\left(1-\frac{u_1}{x}\right)^{\alpha_2-1} \, du_1. \\
  \end{split}
\end{equation}

Setting $u = u_1/x$ (if $x = 0, f_X(x) = 0$, then suppose $x > 0$), we have, 
\begin{equation}
  \begin{split}
    f_X(x) &= \frac{1}{B(\alpha_1, \alpha_2, \alpha_3+\alpha_4)}(1-x)^{\alpha_3+\alpha_4-1} x^{\alpha_1+\alpha_2-1} \int_{0}^{1} u^{\alpha_1-1}(1-u)^{\alpha_2-1} \, du. \\
    &= \frac{1}{B(\alpha_1, \alpha_2, \alpha_3+\alpha_4)}(1-x)^{\alpha_3+\alpha_4-1} x^{\alpha_1+\alpha_2-1} B(\alpha_1, \alpha_2)\\
    &= \frac{1}{B(\alpha_1 + \alpha_2, \alpha_3+\alpha_4)}(1-x)^{\alpha_3+\alpha_4-1} x^{\alpha_1+\alpha_2-1}\\
  \end{split}
\end{equation}
Therefore $X \sim \betadist(\alpha_1+\alpha_2, \alpha_3+\alpha_4)$. Similarly $Y \sim \betadist(\alpha_1+\alpha_3, \alpha_2 + \alpha_4)$.

\end{proof}

\begin{proposition}
  The joint density of $(X,Y)$ with respect to the Lebesgue measure is
  given by 
  \begin{equation}
    f_{X,Y}(x,y) = \frac{1}{B(\boldsymbol{\alpha})}\int_{\Omega} u_1^{\alpha_1 - 1}(x - u_1)^{\alpha_2 -1}(y-u_1)^{\alpha_3-1}(1-x-y+u_1)^{\alpha_4-1} \, du_1,
  \end{equation}
  where 
  $$
  \Omega = (\max(0, x+y-1), \min(x,y)).
  $$
\end{proposition}

\begin{proof}
  Note that
  $$
  \begin{bmatrix}
    U_1 \\ X \\ Y
  \end{bmatrix}  = \begin{bmatrix}
    1 & 0 & 0 \\
    1 & 1 & 0 \\
    1 & 0 & 1
  \end{bmatrix}\begin{bmatrix}
    U_1 \\ U_2 \\ U_3
  \end{bmatrix}, 
  $$
  where the linear function is bijective and differentiable function, such
  that the determinant of the derivative is 1. By the Change of Variable
  Formula, 
  \begin{equation}
    \begin{split}
      f_{U_1,X,Y}(u_1,x,y) &= f_{U_1,U_2,U_3}(u_1, x - u_1, y - u_2) \\ 
      &= \frac{1}{B(\boldsymbol{\alpha})}u_1^{\alpha_1-1}(x-u_1)^{\alpha_2-1}(y-u_1 )^{\alpha_3-1}(1-x-y+u_1)^{\alpha_4-1},
    \end{split}
  \end{equation}
  where $0 \le u_1 \le x, u_1 \le y$, and $0 \le 1 - x - y + u_1$.  
  Hence,
  \begin{equation}
      \label{eq:dist-X-Y}
      f_{X,Y}(x,y) = \frac{1}{B(\boldsymbol{\alpha})}\int_{\Omega} u_1^{\alpha_1-1}(x-u_1)^{\alpha_2-1}(y-u_1)^{\alpha_3-1}(1-x-y+u_1)^{\alpha_4-1} \, du_1,
  \end{equation}
  such that $\Omega = \{u_1 : \max(0, x + y -1) < u_1 < \min(x,y)\}$.
\end{proof}

\begin{proposition}
  The covariance between $X$ and $Y$ is 
  $$\cov(X,Y) = \frac{1}{\tilde{\alpha}^2(\tilde{\alpha}+1)}(\alpha_1\alpha_4 - \alpha_2\alpha_3).$$
\end{proposition}

\begin{proof}

  Let $\tilde{a} = \sum_i \alpha_i$. The covariance between $U_i$ and $U_j$ is \cite[]{lin2016dirichlet} 
\begin{equation}
  \cov(U_i, U_j) = - \frac{\alpha_i\alpha_j}{\tilde{\alpha}^2(\tilde{\alpha}+1)}, i,j = 1,...,4, i \neq j
\end{equation} 
and the variance of $U_i$ is 
\begin{equation}
  \var(U_i) = \frac{\alpha_i(\tilde{\alpha}-\alpha_i)}{\tilde{\alpha}^2(\tilde{\alpha}+1)},
\end{equation}
since $U_i \sim \operatorname{Beta}(\alpha_i, \tilde{\alpha} -\alpha_i)$.
Therefore 
\begin{equation}
  \cov(X,Y) = \cov(U_1+U_2, U_1+U_3) = \frac{1}{\tilde{\alpha}^2(\tilde{\alpha}+1)}(\alpha_1\alpha_4 - \alpha_2\alpha_3)
\end{equation}
  
\end{proof}

The main moments of $X$ and $Y$ are the following 
\begin{align*}
    \ev(X) &= \ev(U_1 + U_2) = \frac{\alpha_1+\alpha_2}{\alpha_1+\alpha_2+\alpha_3+\alpha_4} \\
    \ev(Y) &= \ev(U_1 + U_3) = \frac{\alpha_1+\alpha_3}{\alpha_1+\alpha_2+\alpha_3+\alpha_4} \\
    \var(X) &= \cov(U_1+U_2, U_1+U_2) = \frac{1}{\tilde{\alpha}^2(\tilde{\alpha}+1)}(\alpha_1+\alpha_2)(\alpha_3 + \alpha_4) \\
    \var(Y) &= \cov(U_1+U_3, U_1+U_3) = \frac{1}{\tilde{\alpha}^2(\tilde{\alpha}+1)}(\alpha_1+\alpha_3)(\alpha_2 + \alpha_4)  \\  
    \cor(X,Y) &= \frac{\cov(X,Y)}{\sqrt{\var(X)\var(Y)}} = \frac{\alpha_1\alpha_4 - \alpha_2\alpha_3}{\sqrt{(\alpha_1+\alpha_2)(\alpha_3+\alpha_4)(\alpha_1+\alpha_3)(\alpha_2+\alpha_4)}}
\end{align*}

\begin{remark}
    The original paper has a mistake at page 6. 
\end{remark}

\section{Comments about integration}

The density of $(X,Y)$ with respect to the Lebesgue measure is $f_{X,Y}(x,y)$
as in equation \eqref{eq:dist-X-Y}. Therefore it can be undefined in sets of
null Lebesgue measure in $\R^2$. This section
aims to find them to help writing the function properly. If $\alpha_i \ge 1,
\, i = 1,...,4$, the integral is clearly well defined for every $x,y \in [0,1]$. Let $0 < \alpha_2 = \alpha_3 = a \le 0.5$ and $x = y < 0.5$. Then
\begin{equation*}
  \begin{split}
    f_{X,Y}(x,y) &= \frac{1}{B(\boldsymbol{\alpha})}\int_{0}^x u_1^{\alpha_1-1}(x-u_1)^{a-1}(x-u_1)^{a-1}(1-2x+u_1)^{\alpha_4-1} \, du_1 \\
    &= \frac{1}{B(\boldsymbol{\alpha})}\int_{0}^{x/2} u_1^{\alpha_1-1}(x-u_1)^{2a-2}(1-2x+u_1)^{\alpha_4-1} \, du_1 + \\
    &~~~+ \frac{1}{B(\boldsymbol{\alpha})}\int_{x/2}^x u_1^{\alpha_1-1}(x-u_1)^{2a-2}(1-2x+u_1)^{\alpha_4-1} \, du_1
  \end{split}
\end{equation*}

Note that the first integral is well defined and non-negative. If $\alpha_1 \ge 1$, 
\begin{equation*}
  \begin{split}
    \int_{0}^{x/2} u_1^{\alpha_1-1}&(x-u_1)^{2a-2}(1-2x+u_1)^{\alpha_4-1} \, du_1 \\
    &\le \int_{0}^{x/2} \frac{x}{2}^{\alpha_1-1}\left(\frac{x}{2}\right)^{2a-2}\max\left(\left(1-\frac{3}{2}x\right)^{\alpha_4-1}, (1-2x)^{\alpha_4-1}\right) \, du_1 < +\infty.
  \end{split}
\end{equation*}

If $0 < \alpha_1 < 1$, 
\begin{equation*}
  \begin{split}
    \int_{0}^{x/2} u_1^{\alpha_1-1}&(x-u_1)^{2a-2}(1-2x+u_1)^{\alpha_4-1} \, du_1 \\
    &= \lim_{t\to 0^+}\int_{t}^{x/2} u_1^{\alpha_1-1}\left(\frac{x}{2}\right)^{2a-2}\max\left(\left(1-\frac{3}{2}x\right)^{\alpha_4-1}, (1-2x)^{\alpha_4-1}\right) \, du_1 \\ 
    &= K(x)\lim_{t\to 0^+}\int_{t}^{x/2} u_1^{\alpha_1-1}\, du_1 \\ 
    &= \frac{K(x)}{\alpha_1}\lim_{t\to 0^+} \, \left[\left(\frac{x}{2}\right)^{\alpha_1} - t^{\alpha_1}\right] < +\infty .\\ 
  \end{split}
\end{equation*}
where $K(x)$ is a function of $x$. Moreover, since the integrand is non-negative, so is the integral. On the
other hand, the second integral is not defined: 
\begin{equation*}
  \begin{split}
    \int_{x/2}^{x} u_1^{\alpha_1-1}&(x-u_1)^{2a-2}(1-2x+u_1)^{\alpha_4-1} \, du_1 \\
    &\ge \int_{x/2}^x \min\left(\left(\frac{x}{2}\right)^{\alpha_1-1}, x^{\alpha_1-1}\right)(x-u_1)^{2a-2}\min\left(\left(1-\frac{3}{2}x\right)^{\alpha_4-1}, (1-x)^{\alpha_4-1}\right) \, du_1 \\
    &= K'(x) \int_{0}^{x/2} v^{2a-2} \, dv \\ 
    &= \begin{cases}
      \dfrac{K'(x)}{2a-1} \lim_{t \to 0^+} \left[(x/2)^{2a-1} - t^{2a-1}\right] &\text{ if } a < 0.5 \\ 
      K'(x) \lim_{t \to 0^+} \left[\log(x/2) - \log(t)\right] &\text{ if } a = 0.5
    \end{cases} \\
    &\to +\infty.
  \end{split}
\end{equation*}

Based on this divergence, we conclude that if $0 < \alpha_2 = \alpha_3 \le 0.5$
and $x = y < 0.5$, $f_{X,Y}(x,y)$ is not defined. Note that if $x = y \ge
0.5$, divergence problems still happens, since the problems appear when $u_1$
converges to $x$. Similar calculations show
that if $x + y = 1$ and $0 < \alpha_1 = \alpha_4 \le 0.5$, the density is also
not defined. More generally, $f_{X,Y}(x,y)$ is not defined if 

\begin{enumerate}
  \item $\alpha_1 + \alpha_4 \le 1$ and $x + y = 1$. 
  \item $\alpha_2 + \alpha_3 \le 1$ and $x = y$. 
\end{enumerate}

\section{Specifying parameters
\texorpdfstring{$\boldsymbol{\alpha}$}{alpha}}

Suppose that the researcher has knowledge about the main moments of $X$ and
$Y$, such that $\ev(X) = m_1 \in (0,1), \ev(Y) = m_2 \in (0,1), \var(X) = v_1
\in (0, 1),$ and $\var(Y) =
v_2 \in (0,1)$. Notice that $v_1 + m_1^2 = \var(X_1) + \ev[X_1]^2 = \ev[X_1^2]$ and
$$
\ev[X_1^2] - \ev[X_1] = \frac{(\alpha_1 + \alpha_2 + 1)(\alpha_1 + \alpha_2)}{(\tilde{\alpha} + 1)\tilde{\alpha}} - \frac{\alpha_1 + \alpha_2}{\tilde{\alpha}} = -\frac{(\alpha_1 + \alpha_2)(\alpha_3 + \alpha_4)}{\tilde{\alpha}(\tilde{\alpha}+1)} < 0, 
$$
that is, $v_1 + m_1^2 - m_1 < 0 \implies v_1 < m_1 - m_1^2$ and similarly,
$v_2 < m_2 - m_2^2$. After fixing these quantities, we will have a non-linear system with four equations and four
unknown variables. Hence, we want to solve the following 
\begin{equation}
  \label{eq:system-moments-alpha}
  \begin{cases}
    m_1 = \dfrac{\alpha_1+\alpha_2}{\tilde{\alpha}} \\
    m_2 = \dfrac{\alpha_1+\alpha_3}{\tilde{\alpha}} \\ 
    v_1 = \dfrac{(\alpha_1+\alpha_2)(\alpha_3+\alpha_4)}{\tilde{\alpha}^2(\tilde{\alpha}+1)} = m_1\dfrac{\alpha_3+\alpha_4}{\tilde{\alpha}(\tilde{\alpha}+1)} \\
    v_2 = \dfrac{(\alpha_1+\alpha_3)(\alpha_2+\alpha_4)}{\tilde{\alpha}^2(\tilde{\alpha}+1)} = m_2\dfrac{\alpha_2+\alpha_4}{\tilde{\alpha}(\tilde{\alpha}+1)}.
  \end{cases}
\end{equation}

\begin{proposition}
  System \eqref{eq:system-moments-alpha} has a solution if, and only if, the relation
  \begin{equation}
    \label{eq:v2}
    v_2 = \frac{(1 - m_2)\tilde{\alpha}}{\tilde{\alpha}(\tilde{\alpha}+ 1)} = \frac{1 - m_2}{\frac{m_1 - m_1^2}{v_1}} = \frac{v_1(1 - m_2)}{m_1(1-m_1)},
  \end{equation}
  is satisfied. When there is a solution, there will be
  infinitely many and they all lay in the ray 
  $$
\mathcal{L} = \{(1,-1,-1,1)\alpha_4 + k : \alpha_4 > 0\}, 
$$
such that $k = \left((m_1 + m_2 - 1)\tilde{\alpha}, (1-m_2)\tilde{\alpha},
(1-m_1)\tilde{\alpha}, 0\right)$. 
\end{proposition}

\begin{proof}

The first two equations of the system \eqref{eq:system-moments-alpha} can be
rewritten as a linear system:
\begin{align*}
  (m_1 - 1)\alpha_1 + (m_1 - 1)\alpha_2 + m_1\alpha_3 + m_1\alpha_4 &= 0 \\
  (m_2 - 1)\alpha_1 + m_2\alpha_2 + (m_2-1)\alpha_3 + m_2\alpha_4 &= 0,   
\end{align*}
which is equivalent to 
\begin{align*}
  \alpha_1 + \alpha_2 + \frac{m_1}{m_1-1}\alpha_3 + \frac{m_1}{m_1-1}\alpha_4 &= 0 \\
  \alpha_2 + \frac{1-m_2}{m_1-1}\alpha_3 + \frac{m_1-m_2}{m_1-1}\alpha_4 &= 0.
\end{align*}
Then, we can write $\alpha_1$ and $\alpha_2$ as functions of $\alpha_3$ and
$\alpha_4$:
\begin{align}
  \alpha_1 &= \frac{m_1+m_2-1}{1-m_1}\alpha_3 + \frac{m_2}{1-m_1}\alpha_4 \\
  \alpha_2 &= \frac{1-m_2}{1-m_1}\alpha_3 + \frac{m_1-m_2}{1-m_1}\alpha_4.
\end{align}
With that expression, let $\alpha_1 = a_3\alpha_3 + a_4\alpha_4$ and $\alpha_2
= b_3\alpha_3 + b_4\alpha_4$. Denote $c_3 = a_3 + b_3 + 1$ and $c_4 = a_4 +
b_4 + 1$. Then, consider the third equation of the system
\eqref{eq:system-moments-alpha}, 
\begin{equation*}
  \begin{split}
    &\frac{v_1}{m_1} = \frac{\alpha_3+\alpha_4}{\tilde{\alpha}(\tilde{\alpha} +1)} = \frac{\alpha_3+\alpha_4}{(\alpha_1+\alpha_2+\alpha_3+\alpha_4)^2 + (\alpha_1+\alpha_2+\alpha_3+\alpha_4)} \\
     &\implies \frac{v_1}{m_1}(\alpha_1 + \alpha_2 + \alpha_3 + \alpha_4)^2 = \alpha_3 + \alpha_4 - \frac{v_1}{m_1}(\alpha_1 + \alpha_2 + \alpha_3 + \alpha_4) \\
    &\implies \frac{v_1}{m_1}(c_3\alpha_3 + c_4\alpha_4)^2 = \left(1-\frac{v_1}{m_1}c_3\right)\alpha_3 + \left(1-\frac{v_1}{m_1}c_4\right)\alpha_4 \\
    &\implies \frac{v_1c_3^2}{m_1}\alpha_3^2 + \left(\frac{2v_1c_3c_4\alpha_4+v_1c_3}{m_1} - 1\right)\alpha_3 + \left(\frac{v_1c_4^2\alpha_4^2 + v_1c_4\alpha_4}{m_1} - \alpha_4\right) = 0 \\ 
    &\implies v_1c_3^2\alpha_3^2 + (2v_1c_3c_4\alpha_4+v_1c_3 - m_1)\alpha_3 + (v_1c_4^2\alpha_4^2 + v_1c_4\alpha_4 - m_1\alpha_4) = 0.
  \end{split}
\end{equation*}
Using a Computer Algebra System (CAS) with the Python library SymPy, the above
expression can be simplified as follows:
$$
v_1\alpha_3^2 + \left(v_1(1-m_1) + 2v_1\alpha_4 - m_1(1-m_1)^2\right)\alpha_3 - \alpha_4m_1(1-m_1)^2 + \alpha_4v_1(1 - m_1) + v_1\alpha_4^2 = 0.
$$
This way, the solutions of the above equation are function of $\alpha_4$.
Therefore, after solving the equations, we can use the last equation of the
system \eqref{eq:system-moments-alpha} as a function on of $\alpha_4$. Let, 
$$
\Lambda = \left(v_1(1-m_1) + v_1\alpha_4 - m_1(1-m_1)^2\right).
$$
Then, 
\begin{equation*}
  \begin{split}
    \Delta &= \left(v_1(1-m_1) + 2v_1\alpha_4 - m_1(1-m_1)^2\right)^2 - 4v_1(\alpha_4v_1(1 - m_1) - \alpha_4m_1(1-m_1)^2 + v_1\alpha_4^2), \\
    &= \left(\Lambda + v_1\alpha_4\right)^2 - 4v_1\alpha_4\Lambda \\
    &= \Lambda^2 - 2\Lambda v_1\alpha_4 + (v_1\alpha_4)^2 \\
    &= (\Lambda - v_1\alpha_4)^2 \\
    &= \left(v_1(1-m_1) - m_1(1-m_1)^2\right)^2 \\ 
    &= (1 - m_1)^2(v_1 + m_1^2 - m_1)^2.
  \end{split}
\end{equation*}
Note that $v_1 + m_1^2 = \var(X_1) + \ev[X_1]^2 = \ev[X_1^2]$ and
$$
\ev[X_1^2] - \ev[X_1] = \frac{(\alpha_1 + \alpha_2 + 1)(\alpha_1 + \alpha_2)}{(\tilde{\alpha} + 1)\tilde{\alpha}} - \frac{\alpha_1 + \alpha_2}{\tilde{\alpha}} = -\frac{(\alpha_1 + \alpha_2)(\alpha_3 + \alpha_4)}{\tilde{\alpha}(\tilde{\alpha}+1)} < 0.
$$
Therefore, 
$$
\sqrt{\Delta} = (1-m_1)(m_1 - v_1 - m_1^2)
$$
and 
\begin{equation*}
  \begin{split}
    \alpha_3 &= \frac{1}{2v_1}\left(\left(m_1(1-m_1)^2 - v_1(1-m_1) - 2v_1\alpha_4\right) \pm (1-m_1)(m_1 - v_1 - m_1^2)\right) \\
    &= - \alpha_4 + \frac{(1-m_1)(m_1 - m_1^2 - v_1) \pm (1-m_1)(m_1-v_1-m_1^2)}{2v_1}.
  \end{split}
\end{equation*}
When the sign is negative, we have that $\alpha_3 = - \alpha_4$, an impossible
solution. Then, 
$$
\alpha_3 = \frac{(1-m_1)(m_1 - m_1^2 - v_1)}{v_1} - \alpha_4.
$$

We summarize the expressions in function of $\alpha_4$: 
\begin{align*}
  \alpha_3 &= \frac{(1-m_1)(m_1 - m_1^2 - v_1)}{v_1} - \alpha_4 \\
  \alpha_1 &= \frac{m_1+m_2-1}{1-m_1}\alpha_3 + \frac{m_2}{1-m_1}\alpha_4 = \frac{(m_1 + m_2 - 1)(m_1 - m_1^2 - v_1)}{v_1} + \alpha_4 \\
  \alpha_2 &= \frac{1-m_2}{1-m_1}\alpha_3 + \frac{m_1-m_2}{1-m_1}\alpha_4 = \frac{(1 - m_2)(m_1 - m_1^2 - v_1)}{v_1} - \alpha_4 .
\end{align*}

From here, one can calculate that
$$
\tilde{\alpha} = \frac{m_1 - m_1^2 - v_1}{v_1}.
$$
Since $\alpha_2 + \alpha_4 = (1 - m_2)\tilde{\alpha}$, we have that the last
equation of the system \eqref{eq:system-moments-alpha} is given by
\eqref{eq:v2}, that is, the system $\eqref{eq:system-moments-alpha}$ has a solution if and
only if, equation \eqref{eq:v2} is satisfied. If it is, the solution is the ray 
$$
\mathcal{L} = \{(1,-1,-1,1)\alpha_4 + k : \alpha_4 > 0\}, 
$$
such that $k = \left((m_1 + m_2 - 1)\tilde{\alpha}, (1-m_2)\tilde{\alpha},
(1-m_1)\tilde{\alpha}, 0\right)$. 

\end{proof}


Now change the fourth equation of \eqref{eq:system-moments-alpha} by: 
$$
\cor(X,Y) = \frac{\alpha_1\alpha_4 - \alpha_2\alpha_3}{\sqrt{(\alpha_1+\alpha_2)(\alpha_3+\alpha_4)(\alpha_1+\alpha_3)(\alpha_2+\alpha_4)}} = \frac{\alpha_1\alpha_4 - \alpha_2\alpha_3}{\tilde{\alpha}^2\sqrt{m_1m_2(1-m_1)(1-m_2)}}
$$

Supposing the expression for $\alpha_1, \alpha_2$ and $\alpha_3$, that is,
$m_1, m_2$ and $v_1$ are fixed, and supposing we fix $\rho = \cor(X,Y)$, we
can simplify the above expression (using a software) as follows: 

$$
\rho = \frac{1}{\tilde{\alpha}\sqrt{m_1m_2(1-m_1)(1-m_2)}}\alpha_4 - \sqrt{\frac{(1 - m_1)(1 - m_2)}{m_1m_2}},
$$
which is linear on $\alpha_4$, that is, for fixed values of
$m_1, m_2, v_1$ and $\rho$, there is an unique $\alpha_4$, and hence,
$\alpha_1, \alpha_2$ and $\alpha_3$ that satisfies system
\eqref{eq:system-moments-alpha} with the fourth equation changed by the
correlation. 
